\PassOptionsToPackage{unicode=true}{hyperref} % options for packages loaded elsewhere
\PassOptionsToPackage{hyphens}{url}
\PassOptionsToPackage{dvipsnames,svgnames*,x11names*}{xcolor}
%
\documentclass[
]{article}
\usepackage{lmodern}
\usepackage{amssymb,amsmath}
\usepackage{ifxetex,ifluatex}
\ifnum 0\ifxetex 1\fi\ifluatex 1\fi=0 % if pdftex
  \usepackage[T1]{fontenc}
  \usepackage[utf8]{inputenc}
  \usepackage{textcomp} % provides euro and other symbols
\else % if luatex or xelatex
  \usepackage{unicode-math}
  \defaultfontfeatures{Scale=MatchLowercase}
  \defaultfontfeatures[\rmfamily]{Ligatures=TeX,Scale=1}
\fi
% use upquote if available, for straight quotes in verbatim environments
\IfFileExists{upquote.sty}{\usepackage{upquote}}{}
\IfFileExists{microtype.sty}{% use microtype if available
  \usepackage[]{microtype}
  \UseMicrotypeSet[protrusion]{basicmath} % disable protrusion for tt fonts
}{}
\makeatletter
\@ifundefined{KOMAClassName}{% if non-KOMA class
  \IfFileExists{parskip.sty}{%
    \usepackage{parskip}
  }{% else
    \setlength{\parindent}{0pt}
    \setlength{\parskip}{6pt plus 2pt minus 1pt}}
}{% if KOMA class
  \KOMAoptions{parskip=half}}
\makeatother
\usepackage{xcolor}
\IfFileExists{xurl.sty}{\usepackage{xurl}}{} % add URL line breaks if available
\IfFileExists{bookmark.sty}{\usepackage{bookmark}}{\usepackage{hyperref}}
\hypersetup{
  pdfauthor={María Plaza García},
  colorlinks=true,
  linkcolor=Maroon,
  filecolor=Maroon,
  citecolor=blue,
  urlcolor=blue,
  breaklinks=true}
\urlstyle{same}  % don't use monospace font for urls
\usepackage[margin=2cm]{geometry}
\usepackage{color}
\usepackage{fancyvrb}
\newcommand{\VerbBar}{|}
\newcommand{\VERB}{\Verb[commandchars=\\\{\}]}
\DefineVerbatimEnvironment{Highlighting}{Verbatim}{commandchars=\\\{\}}
% Add ',fontsize=\small' for more characters per line
\usepackage{framed}
\definecolor{shadecolor}{RGB}{248,248,248}
\newenvironment{Shaded}{\begin{snugshade}}{\end{snugshade}}
\newcommand{\AlertTok}[1]{\textcolor[rgb]{0.94,0.16,0.16}{#1}}
\newcommand{\AnnotationTok}[1]{\textcolor[rgb]{0.56,0.35,0.01}{\textbf{\textit{#1}}}}
\newcommand{\AttributeTok}[1]{\textcolor[rgb]{0.77,0.63,0.00}{#1}}
\newcommand{\BaseNTok}[1]{\textcolor[rgb]{0.00,0.00,0.81}{#1}}
\newcommand{\BuiltInTok}[1]{#1}
\newcommand{\CharTok}[1]{\textcolor[rgb]{0.31,0.60,0.02}{#1}}
\newcommand{\CommentTok}[1]{\textcolor[rgb]{0.56,0.35,0.01}{\textit{#1}}}
\newcommand{\CommentVarTok}[1]{\textcolor[rgb]{0.56,0.35,0.01}{\textbf{\textit{#1}}}}
\newcommand{\ConstantTok}[1]{\textcolor[rgb]{0.00,0.00,0.00}{#1}}
\newcommand{\ControlFlowTok}[1]{\textcolor[rgb]{0.13,0.29,0.53}{\textbf{#1}}}
\newcommand{\DataTypeTok}[1]{\textcolor[rgb]{0.13,0.29,0.53}{#1}}
\newcommand{\DecValTok}[1]{\textcolor[rgb]{0.00,0.00,0.81}{#1}}
\newcommand{\DocumentationTok}[1]{\textcolor[rgb]{0.56,0.35,0.01}{\textbf{\textit{#1}}}}
\newcommand{\ErrorTok}[1]{\textcolor[rgb]{0.64,0.00,0.00}{\textbf{#1}}}
\newcommand{\ExtensionTok}[1]{#1}
\newcommand{\FloatTok}[1]{\textcolor[rgb]{0.00,0.00,0.81}{#1}}
\newcommand{\FunctionTok}[1]{\textcolor[rgb]{0.00,0.00,0.00}{#1}}
\newcommand{\ImportTok}[1]{#1}
\newcommand{\InformationTok}[1]{\textcolor[rgb]{0.56,0.35,0.01}{\textbf{\textit{#1}}}}
\newcommand{\KeywordTok}[1]{\textcolor[rgb]{0.13,0.29,0.53}{\textbf{#1}}}
\newcommand{\NormalTok}[1]{#1}
\newcommand{\OperatorTok}[1]{\textcolor[rgb]{0.81,0.36,0.00}{\textbf{#1}}}
\newcommand{\OtherTok}[1]{\textcolor[rgb]{0.56,0.35,0.01}{#1}}
\newcommand{\PreprocessorTok}[1]{\textcolor[rgb]{0.56,0.35,0.01}{\textit{#1}}}
\newcommand{\RegionMarkerTok}[1]{#1}
\newcommand{\SpecialCharTok}[1]{\textcolor[rgb]{0.00,0.00,0.00}{#1}}
\newcommand{\SpecialStringTok}[1]{\textcolor[rgb]{0.31,0.60,0.02}{#1}}
\newcommand{\StringTok}[1]{\textcolor[rgb]{0.31,0.60,0.02}{#1}}
\newcommand{\VariableTok}[1]{\textcolor[rgb]{0.00,0.00,0.00}{#1}}
\newcommand{\VerbatimStringTok}[1]{\textcolor[rgb]{0.31,0.60,0.02}{#1}}
\newcommand{\WarningTok}[1]{\textcolor[rgb]{0.56,0.35,0.01}{\textbf{\textit{#1}}}}
\usepackage{graphicx,grffile}
\makeatletter
\def\maxwidth{\ifdim\Gin@nat@width>\linewidth\linewidth\else\Gin@nat@width\fi}
\def\maxheight{\ifdim\Gin@nat@height>\textheight\textheight\else\Gin@nat@height\fi}
\makeatother
% Scale images if necessary, so that they will not overflow the page
% margins by default, and it is still possible to overwrite the defaults
% using explicit options in \includegraphics[width, height, ...]{}
\setkeys{Gin}{width=\maxwidth,height=\maxheight,keepaspectratio}
\setlength{\emergencystretch}{3em}  % prevent overfull lines
\providecommand{\tightlist}{%
  \setlength{\itemsep}{0pt}\setlength{\parskip}{0pt}}
\setcounter{secnumdepth}{5}
% Redefines (sub)paragraphs to behave more like sections
\ifx\paragraph\undefined\else
  \let\oldparagraph\paragraph
  \renewcommand{\paragraph}[1]{\oldparagraph{#1}\mbox{}}
\fi
\ifx\subparagraph\undefined\else
  \let\oldsubparagraph\subparagraph
  \renewcommand{\subparagraph}[1]{\oldsubparagraph{#1}\mbox{}}
\fi

% set default figure placement to htbp
\makeatletter
\def\fps@figure{htbp}
\makeatother

\usepackage[spanish]{babel}
\usepackage{subfig}

\title{Machine Learning (M0-163)}
\usepackage{etoolbox}
\makeatletter
\providecommand{\subtitle}[1]{% add subtitle to \maketitle
  \apptocmd{\@title}{\par {\large #1 \par}}{}{}
}
\makeatother
\subtitle{Tercera prueba de evaluación continua - Cáncer de mama}
\author{María Plaza García}
\date{12 de Juni, 2020}

\begin{document}
\maketitle

{
\hypersetup{linkcolor=}
\setcounter{tocdepth}{3}
\tableofcontents
}
\textbf{Repositorio Github:}

\url{https://github.com/mariaplaza/PEC3_MachineLearning}

\pagebreak

\hypertarget{introducciuxf3n}{%
\section{Introducción}\label{introducciuxf3n}}

En esta PEC vamos a resolver un análisis para detectar el cáncer en las
mediciones de células biopsiadas de mujeres con masas mamarias
anormales.

Utilizaremos datos de cáncer de mama que incluyen mediciones de imágenes
digitalizadas de aspiración con aguja fina de una masa mamaria. Los
valores representan características de los núcleos celulares presentes
en la imagen digital.

En el fichero BreastCancer1.csv estan los datos sobre el cáncer de mama
de 569 casos de biopsias de cáncer, cada uno con 32 características. La
primera característica es un número de identificación, después son 30
mediciones de laboratorio con valores numéricos y por último, esta el
diagnóstico. El diagnóstico se codifica como M para indicar maligno o B
para indicar benigno.

Los ficheros necesarios para realizar la PEC estan en formato csv. Se
encuentran dentro de mi repositorio github (Github
\protect\hyperlink{ref-github2016github}{2016}), asi como cada uno de
los archivos creados para la realización de esta PEC:

\url{https://github.com/mariaplaza/PEC3_Machine_Learning}

\hypertarget{objetivo}{%
\section{Objetivo}\label{objetivo}}

En esta PEC se analizan estos datos mediante la implementación de los
diferentes algoritmos estudiados: k-Nearest Neighbour, Naive Bayes,
Artificial Neural Network, Support Vector Machine, Árbol de Decisión y
Random Forest para diagnosticar el tipo de cáncer de mama. Para ello se
ha empleado la función de cada algoritmo como se presenta en Lantz
(\protect\hyperlink{ref-lantz2015machine}{2015}) y el paquete
\texttt{caret} en alguno de los casos.

\hypertarget{trabajando-con-los-datos}{%
\section{Trabajando con los datos}\label{trabajando-con-los-datos}}

\hypertarget{lectura-de-datos}{%
\subsection{Lectura de datos}\label{lectura-de-datos}}

Para facilitar la reproducibilidad del informe, se han incluido varios
parámetros en el encabezado \texttt{YAML} del documento cuyos valores se
pueden establecer cuando se procesa el informe. Se ha incluido tanto la
semilla que emplearemos más tarde en la creación de los datos de test y
de entrenamiento así como los nombres de los archivos y la ruta de
acceso, de esta forma podemos leer los datos con el siguiente código:

\begin{Shaded}
\begin{Highlighting}[]
\CommentTok{# Ahora ya se importan los datos a formato data.frame}
\KeywordTok{library}\NormalTok{(readr)}
\NormalTok{m.file <-}\StringTok{ }\KeywordTok{ifelse}\NormalTok{(params}\OperatorTok{$}\NormalTok{folder.data}\OperatorTok{==}\StringTok{""}\NormalTok{, }
\NormalTok{                 params}\OperatorTok{$}\NormalTok{file,}
                 \KeywordTok{file.path}\NormalTok{(params}\OperatorTok{$}\NormalTok{folder.data,params}\OperatorTok{$}\NormalTok{file))}
\NormalTok{data <-}\StringTok{ }\KeywordTok{read.csv}\NormalTok{(}\DataTypeTok{file=}\NormalTok{m.file)}
\end{Highlighting}
\end{Shaded}

\hypertarget{exploraciuxf3n-y-preparaciuxf3n-de-los-datos}{%
\subsection{Exploración y preparación de los
datos}\label{exploraciuxf3n-y-preparaciuxf3n-de-los-datos}}

Comprobamos que el dataset está completo y se presentan los seis
primeros registros de las 6 primeras variables:

\begin{Shaded}
\begin{Highlighting}[]
\KeywordTok{dim}\NormalTok{(data)}
\end{Highlighting}
\end{Shaded}

\begin{verbatim}
[1] 569  32
\end{verbatim}

\begin{Shaded}
\begin{Highlighting}[]
\KeywordTok{head}\NormalTok{(data)[,}\DecValTok{1}\OperatorTok{:}\DecValTok{6}\NormalTok{]}
\end{Highlighting}
\end{Shaded}

\begin{verbatim}
        id radius_mean texture_mean perimeter_mean area_mean smoothness_mean
1   842302       17.99        10.38         122.80    1001.0         0.11840
2   842517       20.57        17.77         132.90    1326.0         0.08474
3 84300903       19.69        21.25         130.00    1203.0         0.10960
4 84348301       11.42        20.38          77.58     386.1         0.14250
5 84358402       20.29        14.34         135.10    1297.0         0.10030
6   843786       12.45        15.70          82.57     477.1         0.12780
\end{verbatim}

Nuestro conjunto de datos, \emph{BreastCancer1.csv}, esta formado por
569 registros, con valores de 32 características.

La primera variable es una variable numérica denominada \texttt{id}. Ya
que es un simple identificador de cada paciente en los datos, no
proporciona infromación útil y necesitamos excluirlo del modelo. Como
está localizado en la primera columna, podemos excluirlo con el
siguiente código, eliminando del \emph{data.frame} la columna número 1.

\begin{Shaded}
\begin{Highlighting}[]
\CommentTok{# eliminamos la característica id}
\NormalTok{data <-}\StringTok{ }\NormalTok{data[}\OperatorTok{-}\DecValTok{1}\NormalTok{]}
\end{Highlighting}
\end{Shaded}

La última variable, ``diagnosis'', es de particular interés, ya que es
el resultado que esperamos predecir. Esta característica indica si la
muestra es de tipo benigno ``B'' o maligno ``M''. La salida
\texttt{table\ ()} indica que las muestras NA son benignas mientras que
NA son malignas. La clasificación existente según la diagnosis:

\begin{Shaded}
\begin{Highlighting}[]
\CommentTok{# tabla de diagnóstico, según la tipología }
\KeywordTok{table}\NormalTok{(data}\OperatorTok{$}\NormalTok{diagnosis)}
\end{Highlighting}
\end{Shaded}

\begin{verbatim}

  B   M 
357 212 
\end{verbatim}

La mayoría de los clasificadores en \emph{Machine Learning} requiren que
esta característica objetivo esté codificada como factor, por lo que
recodificamos la variable, de tipo caracter, en tipo factor.

\begin{Shaded}
\begin{Highlighting}[]
\KeywordTok{class}\NormalTok{(data}\OperatorTok{$}\NormalTok{diagnosis)}
\end{Highlighting}
\end{Shaded}

\begin{verbatim}
[1] "character"
\end{verbatim}

\begin{Shaded}
\begin{Highlighting}[]
\NormalTok{data}\OperatorTok{$}\NormalTok{diagnosis <-}\StringTok{ }\KeywordTok{factor}\NormalTok{(data}\OperatorTok{$}\NormalTok{diagnosis, }\DataTypeTok{levels =} \KeywordTok{c}\NormalTok{(}\StringTok{"B"}\NormalTok{, }\StringTok{"M"}\NormalTok{),}
                         \DataTypeTok{labels =} \KeywordTok{c}\NormalTok{(}\StringTok{"Benigno"}\NormalTok{, }\StringTok{"Maligno"}\NormalTok{))}

\KeywordTok{class}\NormalTok{(data}\OperatorTok{$}\NormalTok{diagnosis)}
\end{Highlighting}
\end{Shaded}

\begin{verbatim}
[1] "factor"
\end{verbatim}

Ahora, cuando miramos la salida \texttt{prop.table\ ()}, notamos que los
valores han sido etiquetados como `Benigno' y `Maligno'.

\begin{Shaded}
\begin{Highlighting}[]
\CommentTok{# tabla de la proporción de cada tipo}
\KeywordTok{round}\NormalTok{(}\KeywordTok{prop.table}\NormalTok{(}\KeywordTok{table}\NormalTok{(data}\OperatorTok{$}\NormalTok{diagnosis)) }\OperatorTok{*}\StringTok{ }\DecValTok{100}\NormalTok{, }\DataTypeTok{digits =} \DecValTok{1}\NormalTok{)}
\end{Highlighting}
\end{Shaded}

\begin{verbatim}

Benigno Maligno 
   62.7    37.3 
\end{verbatim}

Las características restantes de 0 son todas numéricas y, como era de
esperar, consisten en mediciones diferentes de 30 características. Por
ejemplo:

\begin{Shaded}
\begin{Highlighting}[]
\CommentTok{# incluimos tres características numéricas}
\KeywordTok{kable}\NormalTok{(}\KeywordTok{summary}\NormalTok{(data[}\KeywordTok{c}\NormalTok{(}\StringTok{"radius_mean"}\NormalTok{, }\StringTok{"area_mean"}\NormalTok{, }\StringTok{"smoothness_mean"}\NormalTok{)]))}
\end{Highlighting}
\end{Shaded}

\begin{tabular}{l|l|l|l}
\hline
  &  radius\_mean &   area\_mean & smoothness\_mean\\
\hline
 & Min.   : 6.981 & Min.   : 143.5 & Min.   :0.05263\\
\hline
 & 1st Qu.:11.700 & 1st Qu.: 420.3 & 1st Qu.:0.08637\\
\hline
 & Median :13.370 & Median : 551.1 & Median :0.09587\\
\hline
 & Mean   :14.127 & Mean   : 654.9 & Mean   :0.09636\\
\hline
 & 3rd Qu.:15.780 & 3rd Qu.: 782.7 & 3rd Qu.:0.10530\\
\hline
 & Max.   :28.110 & Max.   :2501.0 & Max.   :0.16340\\
\hline
\end{tabular}

La estructura final es:

\begin{Shaded}
\begin{Highlighting}[]
\KeywordTok{str}\NormalTok{(data)}
\end{Highlighting}
\end{Shaded}

\begin{verbatim}
'data.frame':   569 obs. of  31 variables:
 $ radius_mean            : num  18 20.6 19.7 11.4 20.3 ...
 $ texture_mean           : num  10.4 17.8 21.2 20.4 14.3 ...
 $ perimeter_mean         : num  122.8 132.9 130 77.6 135.1 ...
 $ area_mean              : num  1001 1326 1203 386 1297 ...
 $ smoothness_mean        : num  0.1184 0.0847 0.1096 0.1425 0.1003 ...
 $ compactness_mean       : num  0.2776 0.0786 0.1599 0.2839 0.1328 ...
 $ concavity_mean         : num  0.3001 0.0869 0.1974 0.2414 0.198 ...
 $ concave.points_mean    : num  0.1471 0.0702 0.1279 0.1052 0.1043 ...
 $ symmetry_mean          : num  0.242 0.181 0.207 0.26 0.181 ...
 $ fractal_dimension_mean : num  0.0787 0.0567 0.06 0.0974 0.0588 ...
 $ radius_se              : num  1.095 0.543 0.746 0.496 0.757 ...
 $ texture_se             : num  0.905 0.734 0.787 1.156 0.781 ...
 $ perimeter_se           : num  8.59 3.4 4.58 3.44 5.44 ...
 $ area_se                : num  153.4 74.1 94 27.2 94.4 ...
 $ smoothness_se          : num  0.0064 0.00522 0.00615 0.00911 0.01149 ...
 $ compactness_se         : num  0.049 0.0131 0.0401 0.0746 0.0246 ...
 $ concavity_se           : num  0.0537 0.0186 0.0383 0.0566 0.0569 ...
 $ concave.points_se      : num  0.0159 0.0134 0.0206 0.0187 0.0188 ...
 $ symmetry_se            : num  0.03 0.0139 0.0225 0.0596 0.0176 ...
 $ fractal_dimension_se   : num  0.00619 0.00353 0.00457 0.00921 0.00511 ...
 $ radius_worst           : num  25.4 25 23.6 14.9 22.5 ...
 $ texture_worst          : num  17.3 23.4 25.5 26.5 16.7 ...
 $ perimeter_worst        : num  184.6 158.8 152.5 98.9 152.2 ...
 $ area_worst             : num  2019 1956 1709 568 1575 ...
 $ smoothness_worst       : num  0.162 0.124 0.144 0.21 0.137 ...
 $ compactness_worst      : num  0.666 0.187 0.424 0.866 0.205 ...
 $ concavity_worst        : num  0.712 0.242 0.45 0.687 0.4 ...
 $ concave.points_worst   : num  0.265 0.186 0.243 0.258 0.163 ...
 $ symmetry_worst         : num  0.46 0.275 0.361 0.664 0.236 ...
 $ fractal_dimension_worst: num  0.1189 0.089 0.0876 0.173 0.0768 ...
 $ diagnosis              : Factor w/ 2 levels "Benigno","Maligno": 2 2 2 2 2 2 2 2 2 2 ...
\end{verbatim}

Veamos como se comportan las características estudiadas según la
clasificación de nuestra principal variable, disgnosis. Realizamos para
ello una exploración gráfica con un boxplot de las variables numéricas.
Como veremos, se pueden observarlas diferencias en la escala de estas
variables, y para poder distinguir adecuadamente las variables, debemos
cambiar el límite de los ejes:

\begin{Shaded}
\begin{Highlighting}[]
\KeywordTok{boxplot}\NormalTok{(data,}\DataTypeTok{main=}\StringTok{'Datos sin normalizar'}\NormalTok{,}\DataTypeTok{col=}\KeywordTok{c}\NormalTok{(}\StringTok{'lightgreen'}\NormalTok{, }\StringTok{'orange'}\NormalTok{),}\DataTypeTok{cex.axis=}\FloatTok{0.7}\NormalTok{,}
        \DataTypeTok{subset=}\NormalTok{data}\OperatorTok{$}\NormalTok{diagnosis, }\DataTypeTok{ylim=}\KeywordTok{c}\NormalTok{(}\FloatTok{0.05}\NormalTok{,}\DecValTok{1000}\NormalTok{))}

\KeywordTok{boxplot}\NormalTok{(data,}\DataTypeTok{main=}\StringTok{'Datos sin normalizar'}\NormalTok{,}\DataTypeTok{col=}\KeywordTok{c}\NormalTok{(}\StringTok{'lightgreen'}\NormalTok{, }\StringTok{'orange'}\NormalTok{),}\DataTypeTok{cex.axis=}\FloatTok{0.7}\NormalTok{,}
        \DataTypeTok{subset=}\NormalTok{data}\OperatorTok{$}\NormalTok{diagnosis, }\DataTypeTok{ylim=}\KeywordTok{c}\NormalTok{(}\DecValTok{0}\NormalTok{,}\FloatTok{0.30}\NormalTok{))}
\end{Highlighting}
\end{Shaded}

\begin{center}\includegraphics{PEC3_ML_files/figure-latex/unnamed-chunk-8-1} \includegraphics{PEC3_ML_files/figure-latex/unnamed-chunk-8-2} \end{center}

Además se presenta el comportamiento de las primeras 12 variables del
dataset en función del tipo de diagnóstico:

\begin{center}\includegraphics{PEC3_ML_files/figure-latex/unnamed-chunk-9-1} \end{center}

Cada variable muestra diferencias en los resultados según pertenezcan a
tejido sano o tumoral. Sin embargo, aunque normalmente el tejido sano
parece mostrar valores menores que el tejido tumoral, depende
generalmente de la variable. Mostrando por ejemplo valores muy similares
en la viariable ``fractal\_dimension\_mean''. Podemos ver cada variable
de forma independiente, pero ya que tenemos un gran número de variables,
el estudio puede extenderse demasiado.

Como muestra, podemos realizar un anáisis aleatorio de todas las
muestras. Para ello, se toma una muestra de 15 tejidos sanos y otra
muestra de tejidos tumorales, analizando la distribución resultante de 5
variabls aleatorias:

\begin{Shaded}
\begin{Highlighting}[]
\NormalTok{indicesT <-}\StringTok{ }\KeywordTok{which}\NormalTok{(data}\OperatorTok{$}\NormalTok{diagnosis }\OperatorTok{==}\StringTok{'Maligno'}\NormalTok{)}
\NormalTok{indicesN <-}\StringTok{ }\KeywordTok{which}\NormalTok{(data}\OperatorTok{$}\NormalTok{diagnosis }\OperatorTok{==}\StringTok{'Benigno'}\NormalTok{)}

\KeywordTok{set.seed}\NormalTok{(}\DecValTok{123456}\NormalTok{)}
\NormalTok{aleatT <-}\StringTok{ }\KeywordTok{sample}\NormalTok{(indicesT,}\DecValTok{15}\NormalTok{)}
\KeywordTok{set.seed}\NormalTok{(params}\OperatorTok{$}\NormalTok{seed.train)}
\NormalTok{aleatN <-}\StringTok{ }\KeywordTok{sample}\NormalTok{(indicesN,}\DecValTok{15}\NormalTok{)}

\CommentTok{#10 muestras aleatorias}
\KeywordTok{set.seed}\NormalTok{(}\DecValTok{123456}\NormalTok{)}
\NormalTok{aleatG <-}\StringTok{ }\KeywordTok{sample}\NormalTok{(}\DecValTok{1}\OperatorTok{:}\NormalTok{(}\KeywordTok{ncol}\NormalTok{(data)}\OperatorTok{-}\DecValTok{1}\NormalTok{),}\DecValTok{5}\NormalTok{)}

\CommentTok{#Dataset "reducidos" con las muestras}
\NormalTok{datasetT <-}\StringTok{ }\NormalTok{data[aleatT,aleatG]}
\NormalTok{datasetN <-}\StringTok{ }\NormalTok{data[aleatN,aleatG]}

\KeywordTok{par}\NormalTok{(}\DataTypeTok{mfrow=}\KeywordTok{c}\NormalTok{(}\DecValTok{1}\NormalTok{,}\DecValTok{2}\NormalTok{))}
\KeywordTok{boxplot}\NormalTok{(datasetN,}\DataTypeTok{cex.axis=}\FloatTok{0.6}\NormalTok{,}\DataTypeTok{ylab=}\StringTok{''}\NormalTok{,}\DataTypeTok{main=}\StringTok{"Tejido normal"}\NormalTok{,}\DataTypeTok{las=}\DecValTok{2}\NormalTok{, }\DataTypeTok{col=}\StringTok{"lightgreen"}\NormalTok{)}
\KeywordTok{abline}\NormalTok{(}\DataTypeTok{h =} \DecValTok{500}\NormalTok{,}\DataTypeTok{col=}\StringTok{'red'}\NormalTok{)}
\KeywordTok{boxplot}\NormalTok{(datasetT,}\DataTypeTok{cex.axis=}\FloatTok{0.6}\NormalTok{,}\DataTypeTok{ylab=}\StringTok{''}\NormalTok{, }\DataTypeTok{main=}\StringTok{"Tejido tumoral"}\NormalTok{, }\DataTypeTok{las=}\DecValTok{2}\NormalTok{, }\DataTypeTok{col =} \StringTok{"orange"}\NormalTok{)}
\KeywordTok{abline}\NormalTok{(}\DataTypeTok{h =} \DecValTok{500}\NormalTok{,}\DataTypeTok{col=}\StringTok{'red'}\NormalTok{)}
\end{Highlighting}
\end{Shaded}

\begin{center}\includegraphics{PEC3_ML_files/figure-latex/unnamed-chunk-10-1} \end{center}

Por otro lado, podemos realizar una normalización de los datos, de forma
que todos se encuentran entre 0 y 1 y nos permita una visualización más
sencilla y fácil de interpretar:

\begin{Shaded}
\begin{Highlighting}[]
\CommentTok{# Función de normalización}
\NormalTok{normalize <-}\StringTok{ }\ControlFlowTok{function}\NormalTok{(x) \{}
  \KeywordTok{return}\NormalTok{ ((x }\OperatorTok{-}\StringTok{ }\KeywordTok{min}\NormalTok{(x)) }\OperatorTok{/}\StringTok{ }\NormalTok{(}\KeywordTok{max}\NormalTok{(x) }\OperatorTok{-}\StringTok{ }\KeywordTok{min}\NormalTok{(x)))}
\NormalTok{\}}
\end{Highlighting}
\end{Shaded}

Ahora podemos aplicar la función \texttt{normalize()} a las
características numéricas en nuestro marco de datos. En lugar de
normalizar cada una de las 30 variables numéricas individualmente,
utilizaremos una de las funciones de R para automatizar el proceso.

\begin{Shaded}
\begin{Highlighting}[]
\CommentTok{# normalizamos los datos}
\NormalTok{data.norm <-}\StringTok{ }\KeywordTok{as.data.frame}\NormalTok{(}\KeywordTok{lapply}\NormalTok{(data[,}\OperatorTok{-}\DecValTok{31}\NormalTok{], normalize))}
\CommentTok{# Los graficamos}
\KeywordTok{boxplot}\NormalTok{(data.norm,}\DataTypeTok{main=}\StringTok{'Datos normalizados'}\NormalTok{,}\DataTypeTok{col=}\KeywordTok{c}\NormalTok{(}\StringTok{'lightblue'}\NormalTok{, }\StringTok{"orange"}\NormalTok{),}\DataTypeTok{cex.axis=}\FloatTok{0.7}\NormalTok{,}
        \DataTypeTok{subset=}\NormalTok{data}\OperatorTok{$}\NormalTok{diagnosis)}
\KeywordTok{abline}\NormalTok{(}\DataTypeTok{h=}\FloatTok{0.5}\NormalTok{,}\DataTypeTok{lwd=}\DecValTok{1}\NormalTok{, }\DataTypeTok{col=}\StringTok{"red"}\NormalTok{)}
\end{Highlighting}
\end{Shaded}

\begin{center}\includegraphics{PEC3_ML_files/figure-latex/unnamed-chunk-12-1} \end{center}

\hypertarget{particiuxf3n-de-los-datos-en-trainingtest}{%
\subsection{Partición de los datos en
training/test}\label{particiuxf3n-de-los-datos-en-trainingtest}}

Realizamos \emph{aleatoriamente} una extracción de los datos para
entrenar el modelo, en concreto 66.67\% de todas las observaciones, que
son 379 registros, y el resto, 190 registros, para evaluarlo (test).
Usamos los datos sin normalizar y si en algún caso necesitan
normalización o alguna otra transformación lo haremos directamente
dentro del análisis de cada modelo.

\begin{Shaded}
\begin{Highlighting}[]
\CommentTok{#fijamos la semilla para el generador pseudoaleatorio}
\KeywordTok{set.seed}\NormalTok{(params}\OperatorTok{$}\NormalTok{seed.train) }
\CommentTok{# creamos los grupos de entrenamiento (datos.train) y prueba (datos.test) con la selección }
\CommentTok{# aleatoria de los datos asignando el 67% a train y 33% a test (train).}
\NormalTok{train<-}\KeywordTok{sample}\NormalTok{(}\DecValTok{1}\OperatorTok{:}\KeywordTok{nrow}\NormalTok{(data),}\KeywordTok{round}\NormalTok{(}\KeywordTok{nrow}\NormalTok{(data)}\OperatorTok{*}\NormalTok{params}\OperatorTok{$}\NormalTok{p.train,}\DecValTok{0}\NormalTok{))}
\NormalTok{datos.train <-}\StringTok{ }\KeywordTok{data.frame}\NormalTok{(data[train,])}
\NormalTok{datos.test  <-}\StringTok{ }\KeywordTok{data.frame}\NormalTok{(data[}\OperatorTok{-}\NormalTok{train,])}

\KeywordTok{nrow}\NormalTok{(datos.train)}\OperatorTok{/}\KeywordTok{nrow}\NormalTok{(datos.test) }\CommentTok{# debe estar alrededor de 2}
\end{Highlighting}
\end{Shaded}

\begin{verbatim}
[1] 1.994737
\end{verbatim}

El porcentaje de muestras con el diagnóstico de maligno con nuestros
datos de entrenamiento es:

\begin{Shaded}
\begin{Highlighting}[]
\KeywordTok{prop.table}\NormalTok{(}\KeywordTok{table}\NormalTok{(datos.train}\OperatorTok{$}\NormalTok{diagnosis))}
\end{Highlighting}
\end{Shaded}

\begin{verbatim}

  Benigno   Maligno 
0.6385224 0.3614776 
\end{verbatim}

El porcentaje de muestras con el diagnóstico de maligno con nuestros
datos de test es:

\begin{Shaded}
\begin{Highlighting}[]
\KeywordTok{prop.table}\NormalTok{(}\KeywordTok{table}\NormalTok{(datos.test}\OperatorTok{$}\NormalTok{diagnosis))}
\end{Highlighting}
\end{Shaded}

\begin{verbatim}

  Benigno   Maligno 
0.6052632 0.3947368 
\end{verbatim}

Vemos que es muy similar, por lo que los datos están equilibrados.

\hypertarget{aplicaciuxf3n-de-cada-uno-de-los-algoritmos-para-la-clasificaciuxf3n}{%
\section{Aplicación de cada uno de los algoritmos para la
clasificación}\label{aplicaciuxf3n-de-cada-uno-de-los-algoritmos-para-la-clasificaciuxf3n}}

\hypertarget{k-nearest-neighbour}{%
\subsection{k-Nearest Neighbour}\label{k-nearest-neighbour}}

\hypertarget{transformaciuxf3n-de-los-datos}{%
\subsubsection{Transformación de los
datos}\label{transformaciuxf3n-de-los-datos}}

Para aplicar el algoritmo de k-nearest necesitamos tener los datos
normalizados. Ya que se debe aplicar la misma selección de datos
training y test en todos los algoritmos, empleamos la misma selección de
datos anteriores (dentro del vector \emph{train}) pero sobre los datos
normalizados y creamos de nuevo ambos grupos.

Por otro lado, cuando construimos nuestros datos de entrenamiento y
prueba, excluimos la variable objetivo, `diagnosis'. Para entrenar el
modelo kNN, necesitaremos ahora almacenar estas etiquetas de clase en
vectores de factores, divididos en los conjuntos de datos de
entrenamiento y prueba:

\begin{Shaded}
\begin{Highlighting}[]
\CommentTok{#fijamos la semilla para el generador pseudoaleatorio}
\KeywordTok{set.seed}\NormalTok{(params}\OperatorTok{$}\NormalTok{seed.train) }
\CommentTok{# train<-sample(1:nrow(data),round(nrow(data)*params$p.train,0)) # contiene nuestra }
\CommentTok{# selección aleatoria de datos para entrenamiento con el 67% de los datos}
\CommentTok{# creamos los grupos de entrenamiento (train) y prueba (test) con los datos normalizados}
\NormalTok{datos.train.n <-}\StringTok{ }\KeywordTok{data.frame}\NormalTok{(data.norm[train,]) }\CommentTok{# 67% de los datos}
\NormalTok{datos.test.n  <-}\StringTok{ }\KeywordTok{data.frame}\NormalTok{(data.norm[}\OperatorTok{-}\NormalTok{train,]) }\CommentTok{# 33% de los datos}

\KeywordTok{nrow}\NormalTok{(datos.train.n)}\OperatorTok{/}\KeywordTok{nrow}\NormalTok{(datos.test.n) }\CommentTok{# vemos que coincide con la calisifcación realizada }
\end{Highlighting}
\end{Shaded}

\begin{verbatim}
[1] 1.994737
\end{verbatim}

\begin{Shaded}
\begin{Highlighting}[]
\CommentTok{# para los datos sin normalizar. empleamos por tanto la misma selección de datos,}
\CommentTok{# pero normalizados.}

\CommentTok{# Creamos las etiquetas para ambos grupos}
\NormalTok{datos.train.label <-}\StringTok{ }\NormalTok{data[train, }\DecValTok{31}\NormalTok{]}
\NormalTok{datos.test.label <-}\StringTok{ }\NormalTok{data[}\OperatorTok{-}\NormalTok{train, }\DecValTok{31}\NormalTok{]}
\end{Highlighting}
\end{Shaded}

\hypertarget{entrenar-el-modelo}{%
\subsubsection{Entrenar el modelo}\label{entrenar-el-modelo}}

Para clasificar nuestras muestras, usamos la función \texttt{knn()} para
clasificar los datos del grupo test:

\begin{Shaded}
\begin{Highlighting}[]
\KeywordTok{library}\NormalTok{(class)}
\NormalTok{data.test.pred <-}\StringTok{ }\KeywordTok{knn}\NormalTok{(}\DataTypeTok{train =}\NormalTok{ datos.train.n, }\DataTypeTok{test =}\NormalTok{ datos.test.n,}
                      \DataTypeTok{cl =}\NormalTok{ datos.train.label, }\DataTypeTok{k =} \DecValTok{21}\NormalTok{)}
\end{Highlighting}
\end{Shaded}

Como sabemos, la función \texttt{knn()} devuelve un vector factorial de
etiquetas predichas para cada una de las muestras en el conjunto de
datos test, que hemos asignado a \texttt{data.test.pred}.

\hypertarget{predicciuxf3n-y-evaluaciuxf3n-del-algoritmo}{%
\subsubsection{Predicción y Evaluación del
algoritmo}\label{predicciuxf3n-y-evaluaciuxf3n-del-algoritmo}}

El siguiente paso es evaluar qué tan bien las clases predichas en el
vector \texttt{data.test.pred} coinciden con los valores conocidos en el
vector\texttt{datos.test.label}. Para hacer esto, podemos usar la
función \texttt{CrossTable()} en el paquete \texttt{gmodels}:

\begin{Shaded}
\begin{Highlighting}[]
\KeywordTok{library}\NormalTok{(gmodels)}
\CommentTok{# Podemos crear una tabulación cruzada que indique el acuerdo entre los dos vectores, }
\CommentTok{# especificando `prop.chisq = FALSO`}
\NormalTok{conf.mat <-}\StringTok{ }\KeywordTok{CrossTable}\NormalTok{(}\DataTypeTok{x =}\NormalTok{ datos.test.label, }\DataTypeTok{y =}\NormalTok{ data.test.pred,}
           \DataTypeTok{prop.chisq =} \OtherTok{FALSE}\NormalTok{)}
\end{Highlighting}
\end{Shaded}

\begin{verbatim}

 
   Cell Contents
|-------------------------|
|                       N |
|           N / Row Total |
|           N / Col Total |
|         N / Table Total |
|-------------------------|

 
Total Observations in Table:  190 

 
                 | data.test.pred 
datos.test.label |   Benigno |   Maligno | Row Total | 
-----------------|-----------|-----------|-----------|
         Benigno |       114 |         1 |       115 | 
                 |     0.991 |     0.009 |     0.605 | 
                 |     0.919 |     0.015 |           | 
                 |     0.600 |     0.005 |           | 
-----------------|-----------|-----------|-----------|
         Maligno |        10 |        65 |        75 | 
                 |     0.133 |     0.867 |     0.395 | 
                 |     0.081 |     0.985 |           | 
                 |     0.053 |     0.342 |           | 
-----------------|-----------|-----------|-----------|
    Column Total |       124 |        66 |       190 | 
                 |     0.653 |     0.347 |           | 
-----------------|-----------|-----------|-----------|

 
\end{verbatim}

Los casos 114 de 190 indican casos en los que la muestra era benigna, y
el algoritmo kNN la identificó correctamente. Un total de predicciones
65 de 190 fueron verdaderos positivos, es decir, identificados
adecuadamente como ``Malignos''.

Con este modelo se clasificó incorrectamente un total de 10 de las
muestras 190.

Para ver los resultados podemos emplear la función
\texttt{confusionMatrix} del paquete \texttt{caret}, que nos indica los
valores obtnidos de \emph{Sensitivity} y \emph{Specificity}. Es la
función que vamos a emplear en todos los clasificadores para poder
comparalos al final.

\begin{Shaded}
\begin{Highlighting}[]
\KeywordTok{require}\NormalTok{(caret,}\DataTypeTok{quietly =} \OtherTok{TRUE}\NormalTok{)}
\NormalTok{cf.knn <-}\StringTok{ }\KeywordTok{confusionMatrix}\NormalTok{(data.test.pred,datos.test.label,}\DataTypeTok{positive=}\StringTok{"Maligno"}\NormalTok{)}
\NormalTok{cf.knn}
\end{Highlighting}
\end{Shaded}

\begin{verbatim}
Confusion Matrix and Statistics

          Reference
Prediction Benigno Maligno
   Benigno     114      10
   Maligno       1      65
                                          
               Accuracy : 0.9421          
                 95% CI : (0.8988, 0.9707)
    No Information Rate : 0.6053          
    P-Value [Acc > NIR] : < 2e-16         
                                          
                  Kappa : 0.8763          
                                          
 Mcnemar's Test P-Value : 0.01586         
                                          
            Sensitivity : 0.8667          
            Specificity : 0.9913          
         Pos Pred Value : 0.9848          
         Neg Pred Value : 0.9194          
             Prevalence : 0.3947          
         Detection Rate : 0.3421          
   Detection Prevalence : 0.3474          
      Balanced Accuracy : 0.9290          
                                          
       'Positive' Class : Maligno         
                                          
\end{verbatim}

El modelo knn con categoria positiva `Maligno' obtiene una precision de
0.942 y una sensitividad y especificidad de 0.867 y 0.991
respectivamente.

Vamos a intentar ahora otra iteración del modelo para ver si podemos
mejorar el rendimiento y reducir el número de valores que se han
clasificado incorrectamente, especialmente aquellos casos \emph{falsos
negativos}, ya que sería grave identificar como benigno una muestra
cuando en realizad es maligna. cuando en realidad era benigno. Aunque
todos los errores obtenidos deben de tratar evitarse, ya que losw falsos
positivos pueden ocasionar estres al paciente o gastos innecesarios al
hospital.

Para mejorar el modelo vamos a realizar dos variaciones además de
implementar el modelo con el paquete \texttt{caret}. Intentaremos dos
variaciones simples en el clasificador anterior. Primero, un método que
reescala las variables numéricas y en segundo lugar, intentaremos varios
valores del parámetro \emph{k}.

\hypertarget{posible-mejora-1-transformaciuxf3n---estandarizaciuxf3n-de-z}{%
\paragraph{Posible mejora 1: Transformación - estandarización de
z}\label{posible-mejora-1-transformaciuxf3n---estandarizaciuxf3n-de-z}}

Para estandarizar nuestros valores, excepto la variable diagnosis que no
es numérica, emplearemos la función \texttt{scale()} de R, que por
defecto reescala los valores usando the z-score standardization y
almacenamos el resultado en la variable\texttt{data.z}.

\begin{Shaded}
\begin{Highlighting}[]
\CommentTok{# empleamos la función scale() para estandarizar a z-score nuestro data frame}
\NormalTok{data.z <-}\StringTok{ }\KeywordTok{as.data.frame}\NormalTok{(}\KeywordTok{scale}\NormalTok{(data[}\OperatorTok{-}\DecValTok{31}\NormalTok{]))}
\CommentTok{# comprobamos que se ha aplicado la transformación}
\KeywordTok{summary}\NormalTok{(data.z}\OperatorTok{$}\NormalTok{area_mean)}
\end{Highlighting}
\end{Shaded}

\begin{verbatim}
   Min. 1st Qu.  Median    Mean 3rd Qu.    Max. 
-1.4532 -0.6666 -0.2949  0.0000  0.3632  5.2459 
\end{verbatim}

La media de una variable estandarizada z-score siempre debe ser cero
(como ocurre en nuestro caso), y el rango debe ser bastante compacto. Un
z-score mayor que 3 o menor que -3 indica un valor extremadamente raro.
El resumen anterior nos indica que quiza esta transformación no es la
más adecuada, ya que el rango es elevado, más de 3.

Sin embargo, veamos los resultados del modelo. Para ello, volvemos a
entrenar el modelo con los nuevos datos de entrenamiento transformados
usando la función \texttt{knn\ ()}. Luego compararemos las etiquetas
predichas con las etiquetas reales usando \texttt{confusionMatrix()}:

\begin{Shaded}
\begin{Highlighting}[]
\CommentTok{# Creamos de nuevo los grupos de entrenamiento y test con los datos normalizados }
\NormalTok{data.train.z <-}\StringTok{ }\NormalTok{data.z[train, ]}
\NormalTok{data.test.z <-}\StringTok{ }\NormalTok{data.z[}\OperatorTok{-}\NormalTok{train, ]}

\CommentTok{# re-classify test}
\NormalTok{data.test.pred.z <-}\StringTok{ }\KeywordTok{knn}\NormalTok{(}\DataTypeTok{train =}\NormalTok{ data.train.z, }\DataTypeTok{test =}\NormalTok{ data.test.z,}
                      \DataTypeTok{cl =}\NormalTok{ datos.train.label, }\DataTypeTok{k =} \DecValTok{21}\NormalTok{)}
\CommentTok{# Creamos la tabla de resultados}
\NormalTok{cf.knn.z <-}\StringTok{ }\KeywordTok{confusionMatrix}\NormalTok{(data.test.pred.z,datos.test.label,}\DataTypeTok{positive=}\StringTok{"Maligno"}\NormalTok{)}
\NormalTok{cf.knn.z}
\end{Highlighting}
\end{Shaded}

\begin{verbatim}
Confusion Matrix and Statistics

          Reference
Prediction Benigno Maligno
   Benigno     115      12
   Maligno       0      63
                                          
               Accuracy : 0.9368          
                 95% CI : (0.8923, 0.9669)
    No Information Rate : 0.6053          
    P-Value [Acc > NIR] : < 2.2e-16       
                                          
                  Kappa : 0.864           
                                          
 Mcnemar's Test P-Value : 0.001496        
                                          
            Sensitivity : 0.8400          
            Specificity : 1.0000          
         Pos Pred Value : 1.0000          
         Neg Pred Value : 0.9055          
             Prevalence : 0.3947          
         Detection Rate : 0.3316          
   Detection Prevalence : 0.3316          
      Balanced Accuracy : 0.9200          
                                          
       'Positive' Class : Maligno         
                                          
\end{verbatim}

En la tabla anterior los resultados muestran una ligera disminución en
el grado de precisión (Accuracy). Por otro lado, se mejora la
\emph{Specificity} (proporción de sanos correctamente identificados)
pero empeora la \emph{Sensitivity} (detectar la enfermedad en sujetos
enfermos). Por lo que no se mejoran los datos del clasificador anterior.

\hypertarget{posible-mejora-2-distintos-valores-de-k}{%
\paragraph{Posible mejora 2: distintos valores de
k}\label{posible-mejora-2-distintos-valores-de-k}}

Vamos a intentar ahora mejorar el modelo con otros valores de \emph{k},
empleando paa ello los datos normalizados y los datos de prueba, e
imprimimos una tabla resumen con los resultados obtenidosÑ

\begin{Shaded}
\begin{Highlighting}[]
\NormalTok{k_valor <-}\StringTok{ }\NormalTok{params}\OperatorTok{$}\NormalTok{k_value}
\NormalTok{resum <-}\StringTok{ }\KeywordTok{data.frame}\NormalTok{(k_valor, }\DataTypeTok{Accuracy=}\OtherTok{NA}\NormalTok{, }\DataTypeTok{Kappa=}\OtherTok{NA}\NormalTok{, }\DataTypeTok{Sensitivity=}\OtherTok{NA}\NormalTok{, }\DataTypeTok{Specificity=}\OtherTok{NA}\NormalTok{)}
\NormalTok{j <-}\StringTok{ }\DecValTok{0}
\ControlFlowTok{for}\NormalTok{ (i }\ControlFlowTok{in}\NormalTok{ k_valor)\{}
\NormalTok{  j <-}\StringTok{ }\NormalTok{j }\OperatorTok{+}\DecValTok{1}
\NormalTok{  data.test.pred.k <-}\StringTok{ }\KeywordTok{knn}\NormalTok{(}\DataTypeTok{train =}\NormalTok{ datos.train.n, }\DataTypeTok{test =}\NormalTok{ datos.test.n, }\DataTypeTok{cl =}\NormalTok{ datos.train.label, }\DataTypeTok{k=}\NormalTok{i)}
\NormalTok{  cf.knn.k <-}\StringTok{ }\KeywordTok{confusionMatrix}\NormalTok{(data.test.pred.k,datos.test.label,}\DataTypeTok{positive=}\StringTok{"Maligno"}\NormalTok{)}
  
\NormalTok{  resum[j,}\DecValTok{2}\OperatorTok{:}\DecValTok{5}\NormalTok{] <-}\StringTok{ }\KeywordTok{c}\NormalTok{(}\KeywordTok{round}\NormalTok{(cf.knn.k}\OperatorTok{$}\NormalTok{overall[}\StringTok{"Accuracy"}\NormalTok{], }\DecValTok{3}\NormalTok{), }\KeywordTok{round}\NormalTok{(cf.knn.k}\OperatorTok{$}\NormalTok{overall[}\StringTok{"Kappa"}\NormalTok{], }
                                                                  \DecValTok{3}\NormalTok{), }
                    \KeywordTok{round}\NormalTok{(cf.knn.k}\OperatorTok{$}\NormalTok{byClass[}\StringTok{"Sensitivity"}\NormalTok{], }\DecValTok{3}\NormalTok{), }
                    \KeywordTok{round}\NormalTok{(cf.knn.k}\OperatorTok{$}\NormalTok{byClass[}\StringTok{"Specificity"}\NormalTok{], }\DecValTok{3}\NormalTok{))}
\NormalTok{\}}
\KeywordTok{print}\NormalTok{(resum)}
\end{Highlighting}
\end{Shaded}

\begin{verbatim}
  k_valor Accuracy Kappa Sensitivity Specificity
1       1    0.937 0.867       0.907       0.957
2       5    0.947 0.889       0.907       0.974
3      11    0.958 0.910       0.893       1.000
4      17    0.942 0.876       0.867       0.991
5      23    0.942 0.876       0.867       0.991
6      27    0.932 0.853       0.840       0.991
\end{verbatim}

Aunque el primer clasificador empleado no ha sido perfecto, el enfoque
1NN pudo evitar algunos de los falsos negativos a expensas de agregar
falsos positivos.

\hypertarget{naive-bayes}{%
\subsection{Naive Bayes}\label{naive-bayes}}

\hypertarget{transformaciuxf3n-de-los-datos-1}{%
\subsubsection{Transformación de los
datos}\label{transformaciuxf3n-de-los-datos-1}}

Calcula las probabilidades a-posteriores condicionales de una variable
de clase categórica dadas otras variables predictoras independientes
usando la regla de Bayes, sin necesidad de ningún tipo de transformación
en los datos.

\hypertarget{entrenar-el-modelo-1}{%
\subsubsection{Entrenar el modelo}\label{entrenar-el-modelo-1}}

Como ya tenemos los grupos de train y training creados sin los datos
transformados en el apartado 3.3, ya podemos entrenar el algoritmo. El
valor predeterminado del argumento \texttt{laplace} (0) deshabilita el
suavizado de Laplace:

\begin{Shaded}
\begin{Highlighting}[]
\KeywordTok{library}\NormalTok{(e1071)}
\NormalTok{data.nb <-}\StringTok{ }\KeywordTok{naiveBayes}\NormalTok{(datos.train, datos.train.label, }\DataTypeTok{laplace=}\DecValTok{0}\NormalTok{)}
\end{Highlighting}
\end{Shaded}

El resultado del entrenamiento contiene las probabilidades condicionadas
de cada categoria según el tipo de diagnostico. Veamos las cuatro
primeras variables:

\begin{Shaded}
\begin{Highlighting}[]
\NormalTok{data.nb}\OperatorTok{$}\NormalTok{tables[}\DecValTok{1}\OperatorTok{:}\DecValTok{4}\NormalTok{]}
\end{Highlighting}
\end{Shaded}

\begin{verbatim}
$radius_mean
                 radius_mean
datos.train.label     [,1]     [,2]
          Benigno 12.04088 1.744117
          Maligno 17.52124 3.129247

$texture_mean
                 texture_mean
datos.train.label     [,1]     [,2]
          Benigno 18.17752 4.172830
          Maligno 21.81869 3.980472

$perimeter_mean
                 perimeter_mean
datos.train.label      [,1]     [,2]
          Benigno  77.36731 11.64092
          Maligno 115.76584 21.38945

$area_mean
                 area_mean
datos.train.label     [,1]     [,2]
          Benigno 454.3711 129.3648
          Maligno 984.9847 363.7715
\end{verbatim}

\hypertarget{predicciuxf3n-y-evaluaciuxf3n-del-algoritmo-1}{%
\subsubsection{Predicción y Evaluación del
algoritmo}\label{predicciuxf3n-y-evaluaciuxf3n-del-algoritmo-1}}

Aplicamos la función \texttt{predict} del algoritmo con los datos de
test para hacer su predicción:

\begin{Shaded}
\begin{Highlighting}[]
\NormalTok{test.pred.nb <-}\StringTok{ }\KeywordTok{predict}\NormalTok{(data.nb, datos.test)}
\end{Highlighting}
\end{Shaded}

Ahora miramos los resultados en una \emph{Cross Table} usando la función
\texttt{confusionMatrix} del package \texttt{caret}:

\begin{Shaded}
\begin{Highlighting}[]
\KeywordTok{require}\NormalTok{(caret,}\DataTypeTok{quietly =} \OtherTok{TRUE}\NormalTok{)}
\NormalTok{nb.matrix <-}\StringTok{ }\KeywordTok{confusionMatrix}\NormalTok{(test.pred.nb,datos.test.label,}\DataTypeTok{positive=}\StringTok{"Maligno"}\NormalTok{)}
\NormalTok{nb.matrix}
\end{Highlighting}
\end{Shaded}

\begin{verbatim}
Confusion Matrix and Statistics

          Reference
Prediction Benigno Maligno
   Benigno     113       4
   Maligno       2      71
                                          
               Accuracy : 0.9684          
                 95% CI : (0.9325, 0.9883)
    No Information Rate : 0.6053          
    P-Value [Acc > NIR] : <2e-16          
                                          
                  Kappa : 0.9336          
                                          
 Mcnemar's Test P-Value : 0.6831          
                                          
            Sensitivity : 0.9467          
            Specificity : 0.9826          
         Pos Pred Value : 0.9726          
         Neg Pred Value : 0.9658          
             Prevalence : 0.3947          
         Detection Rate : 0.3737          
   Detection Prevalence : 0.3842          
      Balanced Accuracy : 0.9646          
                                          
       'Positive' Class : Maligno         
                                          
\end{verbatim}

El modelo Naive Bayes con categoria positiva `Maligno' obtiene una
precisión de 0.968 y una sensitividad y especificidad de 0.947 y 0.983
respectivamente.

\hypertarget{posible-mejora-laplace-1}{%
\paragraph{\texorpdfstring{Posible mejora: \emph{laplace =
1}}{Posible mejora: laplace = 1}}\label{posible-mejora-laplace-1}}

Ahora se prueba a entrenar el modelo aplicando la opción
\texttt{laplace\ =\ 1}:

\begin{Shaded}
\begin{Highlighting}[]
\NormalTok{data.nb2 <-}\StringTok{ }\KeywordTok{naiveBayes}\NormalTok{(datos.train, datos.train.label, }\DataTypeTok{laplace=}\DecValTok{1}\NormalTok{)}
\end{Highlighting}
\end{Shaded}

y se hace la predicción:

\begin{Shaded}
\begin{Highlighting}[]
\NormalTok{test.pred.nb2 <-}\StringTok{ }\KeywordTok{predict}\NormalTok{(data.nb2, datos.test)}
\end{Highlighting}
\end{Shaded}

Ahora se evalua el modelo con la función \texttt{confusionMatrix} del
package \texttt{caret}.

\begin{Shaded}
\begin{Highlighting}[]
\NormalTok{nb.matrix}\FloatTok{.1}\NormalTok{ <-}\StringTok{ }\KeywordTok{confusionMatrix}\NormalTok{(test.pred.nb2,datos.test.label,}\DataTypeTok{positive=}\StringTok{"Maligno"}\NormalTok{)}
\end{Highlighting}
\end{Shaded}

El nuevo modelo daplicando la opción \texttt{laplace\ =\ 1} obtiene una
precision de 0.963 y una sensitividad y especificidad de 0.933 y 0.983
respectivamente. Vemos que el modelo obtenido con SVM lineal tiene una
mayor precision

\hypertarget{curvas-roc}{%
\paragraph{Curvas ROC}\label{curvas-roc}}

Se presentan las curvas ROC para el modelo de Naive Bayes con
\texttt{laplace=0} y \texttt{laplace=\ 1}.

\textbf{Caso \texttt{laplace=0}}

El primer paso es obtener las probabilidades diagnosis (Maligno/Benigno)
para cada muestra de los datos de test:

\begin{Shaded}
\begin{Highlighting}[]
\NormalTok{test.pred.nb3 <-}\StringTok{ }\KeywordTok{predict}\NormalTok{(data.nb, datos.test, }\DataTypeTok{type=}\StringTok{"raw"}\NormalTok{)}
\KeywordTok{tail}\NormalTok{(test.pred.nb3)}
\end{Highlighting}
\end{Shaded}

\begin{verbatim}
             Benigno      Maligno
[185,]  1.000000e+00 1.774260e-19
[186,]  1.000000e+00 2.401845e-16
[187,]  1.000000e+00 4.743641e-19
[188,]  1.000000e+00 5.832161e-15
[189,]  6.419662e-45 1.000000e+00
[190,] 2.146303e-167 1.000000e+00
\end{verbatim}

Con la información de las probabilidades de la clase positiva
(``Maligno'' o 2) se construye la curva ROC.

\begin{Shaded}
\begin{Highlighting}[]
\KeywordTok{require}\NormalTok{(ROCR,}\DataTypeTok{quietly=}\OtherTok{TRUE}\NormalTok{)}

\NormalTok{pred <-}\StringTok{ }\KeywordTok{prediction}\NormalTok{(}\DataTypeTok{predictions=}\NormalTok{ test.pred.nb3[,}\DecValTok{2}\NormalTok{], }\DataTypeTok{labels=}\NormalTok{datos.test.label)}
\NormalTok{perf <-}\StringTok{ }\KeywordTok{performance}\NormalTok{(pred, }\DataTypeTok{measure=}\StringTok{"tpr"}\NormalTok{, }\DataTypeTok{x.measure=}\StringTok{"fpr"}\NormalTok{)}

\KeywordTok{plot}\NormalTok{(perf, }\DataTypeTok{main=} \StringTok{"ROC curve Laplace 0"}\NormalTok{, }\DataTypeTok{col=} \StringTok{"blue"}\NormalTok{, }\DataTypeTok{lwd=}\DecValTok{3}\NormalTok{, }\DataTypeTok{colorize=}\OtherTok{TRUE}\NormalTok{)}
\KeywordTok{abline}\NormalTok{(}\DataTypeTok{a=}\DecValTok{0}\NormalTok{, }\DataTypeTok{b=} \DecValTok{1}\NormalTok{, }\DataTypeTok{lwd=} \DecValTok{2}\NormalTok{, }\DataTypeTok{lty =} \DecValTok{2}\NormalTok{)}
\NormalTok{perf.auc <-}\StringTok{ }\KeywordTok{performance}\NormalTok{(pred, }\DataTypeTok{measure =}\StringTok{"auc"}\NormalTok{)}
\end{Highlighting}
\end{Shaded}

\begin{center}\includegraphics{PEC3_ML_files/figure-latex/unnamed-chunk-31-1} \end{center}

El area bajo la curva es \textbf{0.996058}.

\textbf{Caso \texttt{laplace=1}}

El primer paso es obtener las probabilidades diagnosis (Maligno/Benigno)
para cada muestra de los datos de test:

\begin{Shaded}
\begin{Highlighting}[]
\NormalTok{test.pred.nb4 <-}\StringTok{ }\KeywordTok{predict}\NormalTok{(data.nb2, datos.test, }\DataTypeTok{type=}\StringTok{"raw"}\NormalTok{)}
\KeywordTok{tail}\NormalTok{(test.pred.nb4)}
\end{Highlighting}
\end{Shaded}

\begin{verbatim}
             Benigno      Maligno
[185,]  1.000000e+00 1.281699e-18
[186,]  1.000000e+00 1.735057e-15
[187,]  1.000000e+00 3.426735e-18
[188,]  1.000000e+00 4.213066e-14
[189,]  2.650074e-44 1.000000e+00
[190,] 8.860065e-167 1.000000e+00
\end{verbatim}

Con la información de las probabilidades de la clase positiva (1) se
construye la curva ROC.

\begin{Shaded}
\begin{Highlighting}[]
\NormalTok{pred2 <-}\StringTok{ }\KeywordTok{prediction}\NormalTok{(}\DataTypeTok{predictions=}\NormalTok{ test.pred.nb4[,}\DecValTok{2}\NormalTok{], }\DataTypeTok{labels=}\NormalTok{datos.test.label)}
\NormalTok{perf2 <-}\StringTok{ }\KeywordTok{performance}\NormalTok{(pred2, }\DataTypeTok{measure=}\StringTok{"tpr"}\NormalTok{, }\DataTypeTok{x.measure=}\StringTok{"fpr"}\NormalTok{)}

\KeywordTok{plot}\NormalTok{(perf2, }\DataTypeTok{main=} \StringTok{"ROC curve Laplace 1"}\NormalTok{, }\DataTypeTok{col=} \StringTok{"blue"}\NormalTok{, }\DataTypeTok{lwd=}\DecValTok{3}\NormalTok{, }\DataTypeTok{colorize=}\OtherTok{TRUE}\NormalTok{)}
\KeywordTok{abline}\NormalTok{(}\DataTypeTok{a=}\DecValTok{0}\NormalTok{, }\DataTypeTok{b=} \DecValTok{1}\NormalTok{, }\DataTypeTok{lwd=} \DecValTok{2}\NormalTok{, }\DataTypeTok{lty =} \DecValTok{2}\NormalTok{)}
\NormalTok{perf2.auc <-}\StringTok{ }\KeywordTok{performance}\NormalTok{(pred2, }\DataTypeTok{measure =}\StringTok{"auc"}\NormalTok{)}

\KeywordTok{unlist}\NormalTok{(perf2.auc}\OperatorTok{@}\NormalTok{y.values)}
\end{Highlighting}
\end{Shaded}

\begin{verbatim}
[1] 0.9951304
\end{verbatim}

\begin{center}\includegraphics{PEC3_ML_files/figure-latex/unnamed-chunk-33-1} \end{center}

El area bajo la curva es \textbf{0.9951304} y con \texttt{laplace\ =\ 0}
es \textbf{\texttt{unlist(perf.auc@y.values)}}. Como vemos no mejora la
predicción, por lo que no tiene sentido cambiar el argumento de 0 a 1.

\hypertarget{artificial-neural-network}{%
\subsection{Artificial Neural Network}\label{artificial-neural-network}}

\hypertarget{transformaciuxf3n-de-los-datos-2}{%
\subsubsection{Transformación de los
datos}\label{transformaciuxf3n-de-los-datos-2}}

Hay que normalizar las variables para que tomen valores entre 0 y 1, tal
y como hemos realizado en la clasificación de k-nn.~Los datos
normalizados se encuentran dentro de \texttt{data.norm}

\begin{Shaded}
\begin{Highlighting}[]
\CommentTok{#data.norm <- as.data.frame(lapply(data[,-31], normalize))}
\end{Highlighting}
\end{Shaded}

Se confirma que el rango de valores esta entre 0 y 1.

\begin{Shaded}
\begin{Highlighting}[]
\KeywordTok{summary}\NormalTok{(data.norm)[,}\DecValTok{1}\OperatorTok{:}\DecValTok{5}\NormalTok{]}
\end{Highlighting}
\end{Shaded}

\begin{verbatim}
  radius_mean      texture_mean    perimeter_mean     area_mean      smoothness_mean 
 Min.   :0.0000   Min.   :0.0000   Min.   :0.0000   Min.   :0.0000   Min.   :0.0000  
 1st Qu.:0.2233   1st Qu.:0.2185   1st Qu.:0.2168   1st Qu.:0.1174   1st Qu.:0.3046  
 Median :0.3024   Median :0.3088   Median :0.2933   Median :0.1729   Median :0.3904  
 Mean   :0.3382   Mean   :0.3240   Mean   :0.3329   Mean   :0.2169   Mean   :0.3948  
 3rd Qu.:0.4164   3rd Qu.:0.4089   3rd Qu.:0.4168   3rd Qu.:0.2711   3rd Qu.:0.4755  
 Max.   :1.0000   Max.   :1.0000   Max.   :1.0000   Max.   :1.0000   Max.   :1.0000  
\end{verbatim}

Ahora se crean tantas variables binarias como categorias tiene la
variable \texttt{diagnosis}. Por lo que ahora nuestra matriz de datos
tiene 32 variables, las 30 variables numéricas, y las dos variables que
creamos que son de tipo binario, con valores de TRUE y/o FALSE.

\begin{Shaded}
\begin{Highlighting}[]
\CommentTok{# Creación de variables binarias en lugar de usar la variable factor}
\NormalTok{data.norm}\OperatorTok{$}\NormalTok{M <-}\StringTok{ }\NormalTok{data}\OperatorTok{$}\NormalTok{diagnosis}\OperatorTok{==}\StringTok{"Maligno"}
\NormalTok{data.norm}\OperatorTok{$}\NormalTok{B <-}\StringTok{ }\NormalTok{data}\OperatorTok{$}\NormalTok{diagnosis}\OperatorTok{==}\StringTok{"Benigno"}
\end{Highlighting}
\end{Shaded}

Del mismo modo emplearemos los conjuntos para entrenamiento y prueba
creados en el apartado anterior con datos normalizados, pero incluyendo
ahora las dos nuevas variables:

\begin{Shaded}
\begin{Highlighting}[]
\CommentTok{# creamos las etiquetas para los grupos de training and test}
\NormalTok{datos.train.nn <-}\StringTok{ }\KeywordTok{data.frame}\NormalTok{(data.norm[train,])}
\NormalTok{datos.test.nn  <-}\StringTok{ }\KeywordTok{data.frame}\NormalTok{(data.norm[}\OperatorTok{-}\NormalTok{train,])}

\NormalTok{datos.train.label <-}\StringTok{ }\NormalTok{data[train, }\DecValTok{31}\NormalTok{]}
\NormalTok{datos.test.label <-}\StringTok{ }\NormalTok{data[}\OperatorTok{-}\NormalTok{train, }\DecValTok{31}\NormalTok{]}
\end{Highlighting}
\end{Shaded}

\hypertarget{entrenar-el-modelo-2}{%
\subsubsection{Entrenar el modelo}\label{entrenar-el-modelo-2}}

Para la construcción de la red neuronal artificial se usa la función
\texttt{neuralnet()} del paquete \emph{neuralnet}. La fórmula del modelo
tiene 32 nodos de entrada y 2 (niveles de \texttt{data\$diagnosis})
nodos de salida:

\begin{Shaded}
\begin{Highlighting}[]
\CommentTok{# Creamos una formula para desarrollar el modelo }
\CommentTok{# con todas y cada una de las 32 variables:}
\NormalTok{xnam <-}\StringTok{ }\KeywordTok{names}\NormalTok{(data[}\DecValTok{1}\OperatorTok{:}\DecValTok{30}\NormalTok{])}
\NormalTok{(fmla <-}\StringTok{ }\KeywordTok{as.formula}\NormalTok{(}\KeywordTok{paste}\NormalTok{(}\StringTok{"M + B ~ "}\NormalTok{, }\KeywordTok{paste}\NormalTok{(xnam, }\DataTypeTok{collapse=} \StringTok{"+"}\NormalTok{))))}
\end{Highlighting}
\end{Shaded}

\begin{verbatim}
M + B ~ radius_mean + texture_mean + perimeter_mean + area_mean + 
    smoothness_mean + compactness_mean + concavity_mean + concave.points_mean + 
    symmetry_mean + fractal_dimension_mean + radius_se + texture_se + 
    perimeter_se + area_se + smoothness_se + compactness_se + 
    concavity_se + concave.points_se + symmetry_se + fractal_dimension_se + 
    radius_worst + texture_worst + perimeter_worst + area_worst + 
    smoothness_worst + compactness_worst + concavity_worst + 
    concave.points_worst + symmetry_worst + fractal_dimension_worst
\end{verbatim}

El modelo aplicado es de un nodo en la capa oculta, esto se consigue con
el argumento \texttt{hidden=1}. El modelo se construye con el argumento
\texttt{linear.output=FALSE} ya que se trata de un problema de
clasificación.

\begin{Shaded}
\begin{Highlighting}[]
\CommentTok{# simple ANN con una única neurona en la capa oculta}
\KeywordTok{set.seed}\NormalTok{(params}\OperatorTok{$}\NormalTok{seed.clsfier)  }\CommentTok{# semilla que nos garantiza resultados reproducibles}
\KeywordTok{library}\NormalTok{(neuralnet)}
\NormalTok{nn.model}\FloatTok{.1}\NormalTok{ <-}\StringTok{ }\KeywordTok{neuralnet}\NormalTok{(fmla, }\DataTypeTok{data =}\NormalTok{ datos.train.nn, }\DataTypeTok{hidden =} \DecValTok{1}\NormalTok{, }\DataTypeTok{linear.output =} \OtherTok{FALSE}\NormalTok{)}
\CommentTok{# Visualizamos la topología de la red neuronal:}
\KeywordTok{plot}\NormalTok{(nn.model}\FloatTok{.1}\NormalTok{, }\DataTypeTok{rep =} \StringTok{"best"}\NormalTok{, }\DataTypeTok{main =} \StringTok{"ANN con un nodo"}\NormalTok{)}
\end{Highlighting}
\end{Shaded}

\begin{center}\includegraphics{PEC3_ML_files/figure-latex/unnamed-chunk-38-1} \end{center}

\hypertarget{predicciuxf3n-y-evaluaciuxf3n-del-algoritmo-2}{%
\subsubsection{Predicción y Evaluación del
algoritmo}\label{predicciuxf3n-y-evaluaciuxf3n-del-algoritmo-2}}

Una vez obtenido el primer modelo, se evalua su rendimiento con los
datos de test. Se debe de clasificar las muestras test con la función
\texttt{predict} al igual que en los algoritmos anteriores. El resultado
de la matriz de confusión con los datos de test en este modelo podemos
obtenerla con el siguiente código:

\begin{Shaded}
\begin{Highlighting}[]
\NormalTok{nn.model.}\FloatTok{1.}\NormalTok{matrix <-}\StringTok{ }\KeywordTok{predict}\NormalTok{(nn.model}\FloatTok{.1}\NormalTok{, datos.test.nn[, }\DecValTok{1}\OperatorTok{:}\DecValTok{32}\NormalTok{])}

\CommentTok{# Creamos una salida binaria múltiple a nuestra salida categórica}
\NormalTok{max.idx <-}\StringTok{ }\ControlFlowTok{function}\NormalTok{(xxy) \{}
    \KeywordTok{return}\NormalTok{(}\KeywordTok{which}\NormalTok{(xxy }\OperatorTok{==}\StringTok{ }\KeywordTok{max}\NormalTok{(xxy)))}
\NormalTok{\}}
\NormalTok{idx1 <-}\StringTok{ }\KeywordTok{apply}\NormalTok{(nn.model.}\FloatTok{1.}\NormalTok{matrix, }\DecValTok{1}\NormalTok{, max.idx)}
\NormalTok{prediction <-}\StringTok{ }\KeywordTok{c}\NormalTok{(}\StringTok{"Maligno"}\NormalTok{, }\StringTok{"Benigno"}\NormalTok{)[idx1]}
\NormalTok{result <-}\StringTok{ }\KeywordTok{table}\NormalTok{(prediction, data}\OperatorTok{$}\NormalTok{diagnosis[}\OperatorTok{-}\NormalTok{train])  }\CommentTok{# comparar con los datos test}

\CommentTok{# Resultados}
\KeywordTok{library}\NormalTok{(caret)}
\NormalTok{(cmatrix1 <-}\StringTok{ }\KeywordTok{confusionMatrix}\NormalTok{(result, }\DataTypeTok{positive =} \StringTok{"Maligno"}\NormalTok{))}
\end{Highlighting}
\end{Shaded}

\begin{verbatim}
Confusion Matrix and Statistics

          
prediction Benigno Maligno
   Benigno     107       4
   Maligno       8      71
                                          
               Accuracy : 0.9368          
                 95% CI : (0.8923, 0.9669)
    No Information Rate : 0.6053          
    P-Value [Acc > NIR] : <2e-16          
                                          
                  Kappa : 0.869           
                                          
 Mcnemar's Test P-Value : 0.3865          
                                          
            Sensitivity : 0.9467          
            Specificity : 0.9304          
         Pos Pred Value : 0.8987          
         Neg Pred Value : 0.9640          
             Prevalence : 0.3947          
         Detection Rate : 0.3737          
   Detection Prevalence : 0.4158          
      Balanced Accuracy : 0.9386          
                                          
       'Positive' Class : Maligno         
                                          
\end{verbatim}

El modelo de una capa con categoria positiva `Maligno' obtiene una
precision de 0.937 y una sensitividad y especificidad de 0.947 y 0.93
respectivamente.

\hypertarget{posible-mejora-3-nodos}{%
\paragraph{Posible mejora: 3 nodos}\label{posible-mejora-3-nodos}}

Se plantean \emph{3 nodos} en la capa oculta para tratar de mejorar el
rendimiento.

\begin{Shaded}
\begin{Highlighting}[]
\CommentTok{# ANN simple con tres neuronas ocultas}
\KeywordTok{set.seed}\NormalTok{(params}\OperatorTok{$}\NormalTok{seed.clsfier)  }\CommentTok{# garantiza los resultados reproducibles}
\NormalTok{nn.model}\FloatTok{.3}\NormalTok{ <-}\StringTok{ }\KeywordTok{neuralnet}\NormalTok{(fmla, }\DataTypeTok{data =}\NormalTok{ datos.train.nn, }\DataTypeTok{hidden =} \DecValTok{3}\NormalTok{, }\DataTypeTok{linear.output =} \OtherTok{FALSE}\NormalTok{)}
\CommentTok{# visualizamos la topología de la red}
\KeywordTok{plot}\NormalTok{(nn.model}\FloatTok{.3}\NormalTok{, }\DataTypeTok{rep =} \StringTok{"best"}\NormalTok{, }\DataTypeTok{main =} \StringTok{"ANN con tres nodos"}\NormalTok{)}
\end{Highlighting}
\end{Shaded}

\begin{center}\includegraphics{PEC3_ML_files/figure-latex/unnamed-chunk-40-1} \end{center}

El resultado de la matriz de confusión con los datos de test es:

\begin{Shaded}
\begin{Highlighting}[]
\NormalTok{nn.model.}\FloatTok{3.}\NormalTok{matrix <-}\StringTok{ }\KeywordTok{predict}\NormalTok{(nn.model}\FloatTok{.3}\NormalTok{, datos.test.nn[, }\DecValTok{1}\OperatorTok{:}\DecValTok{32}\NormalTok{])}

\NormalTok{idx3 <-}\StringTok{ }\KeywordTok{apply}\NormalTok{(nn.model.}\FloatTok{3.}\NormalTok{matrix, }\DecValTok{1}\NormalTok{, max.idx)}
\NormalTok{prediction3 <-}\StringTok{ }\KeywordTok{c}\NormalTok{(}\StringTok{"Maligno"}\NormalTok{, }\StringTok{"Benigno"}\NormalTok{)[idx3]}
\NormalTok{result}\FloatTok{.3}\NormalTok{ <-}\StringTok{ }\KeywordTok{table}\NormalTok{(prediction3, data}\OperatorTok{$}\NormalTok{diagnosis[}\OperatorTok{-}\NormalTok{train])  }\CommentTok{# comparar con los datos test}

\CommentTok{# Resultados}
\NormalTok{(cmatrix3 <-}\StringTok{ }\KeywordTok{confusionMatrix}\NormalTok{(result}\FloatTok{.3}\NormalTok{, }\DataTypeTok{positive =} \StringTok{"Maligno"}\NormalTok{))}
\end{Highlighting}
\end{Shaded}

\begin{verbatim}
Confusion Matrix and Statistics

           
prediction3 Benigno Maligno
    Benigno     106       5
    Maligno       9      70
                                          
               Accuracy : 0.9263          
                 95% CI : (0.8795, 0.9591)
    No Information Rate : 0.6053          
    P-Value [Acc > NIR] : <2e-16          
                                          
                  Kappa : 0.8472          
                                          
 Mcnemar's Test P-Value : 0.4227          
                                          
            Sensitivity : 0.9333          
            Specificity : 0.9217          
         Pos Pred Value : 0.8861          
         Neg Pred Value : 0.9550          
             Prevalence : 0.3947          
         Detection Rate : 0.3684          
   Detection Prevalence : 0.4158          
      Balanced Accuracy : 0.9275          
                                          
       'Positive' Class : Maligno         
                                          
\end{verbatim}

El nuevo modelo con 3 nodos ocultos obtiene una precisión de 0.926 y una
sensitividad y especificidad de 0.933 y 0.922 respectivamente. Vemos que
el modelo obtenido con un solo nodo tiene una mayor precision.

Si se parecen debemos emplear el modelo más sencillo para evitar
overfitting.

\hypertarget{paquete-caret-modelo-nnet}{%
\subsubsection{\texorpdfstring{Paquete \emph{caret}: modelo
\texttt{nnet}}{Paquete caret: modelo nnet}}\label{paquete-caret-modelo-nnet}}

La función \texttt{nnet} admite datos de tipo factor, así que no hay que
transformar la variable \texttt{diagnosis} en variables binarias.
Empleamos los datos de entrenamiento y test creados en el apartado 3.3.
Veamos como responde el modelo con los datos de test y train:

\begin{Shaded}
\begin{Highlighting}[]
\CommentTok{# modelo Train/test sin repetición}
\KeywordTok{library}\NormalTok{(caret)}
\KeywordTok{library}\NormalTok{(NeuralNetTools)}
\NormalTok{model.nnet <-}\StringTok{ }\KeywordTok{train}\NormalTok{(diagnosis }\OperatorTok{~}\StringTok{ }\NormalTok{., datos.train, }\DataTypeTok{method=}\StringTok{'nnet'}\NormalTok{, }
               \DataTypeTok{trControl=} \KeywordTok{trainControl}\NormalTok{(}\DataTypeTok{method=}\StringTok{'none'}\NormalTok{), }
               \DataTypeTok{preProcess =} \StringTok{"range"}\NormalTok{, }\CommentTok{# probamos con un preproceso de los datos}
               \DataTypeTok{tuneGrid=} \OtherTok{NULL}\NormalTok{, }\DataTypeTok{tuneLength=}\DecValTok{1}\NormalTok{ ,}\DataTypeTok{trace =} \OtherTok{FALSE}\NormalTok{) }\CommentTok{#}

\KeywordTok{plotnet}\NormalTok{(model.nnet, }\DataTypeTok{main=}\StringTok{"modelo `nnet` con caret"}\NormalTok{)}
\CommentTok{#summary(model.nnet)}
\NormalTok{prediction.nnet <-}\StringTok{ }\KeywordTok{predict}\NormalTok{(model.nnet, datos.test)         }\CommentTok{# predecir los resultados}
\NormalTok{result.caret1 <-}\StringTok{ }\KeywordTok{table}\NormalTok{(prediction.nnet, datos.test}\OperatorTok{$}\NormalTok{diagnosis) }\CommentTok{# comparar con los datos test}
\NormalTok{result.caret1}
\end{Highlighting}
\end{Shaded}

\begin{verbatim}
               
prediction.nnet Benigno Maligno
        Benigno     108       6
        Maligno       7      69
\end{verbatim}

\begin{Shaded}
\begin{Highlighting}[]
\CommentTok{# la función predict también puede devolver la probabilidad para cada clase:}
\NormalTok{prediction.nnet1 <-}\StringTok{ }\KeywordTok{predict}\NormalTok{(model.nnet, datos.test, }\DataTypeTok{type=}\StringTok{"prob"}\NormalTok{)  }
\KeywordTok{head}\NormalTok{(prediction.nnet1)}
\end{Highlighting}
\end{Shaded}

\begin{verbatim}
   Benigno Maligno
2        0       1
4        0       1
5        0       1
9        0       1
11       0       1
19       0       1
\end{verbatim}

\begin{Shaded}
\begin{Highlighting}[]
\CommentTok{# Resultados}
\NormalTok{(cmatrix.nnet <-}\StringTok{ }\KeywordTok{confusionMatrix}\NormalTok{(result.caret1,}\DataTypeTok{positive =} \StringTok{"Maligno"}\NormalTok{))}
\end{Highlighting}
\end{Shaded}

\begin{verbatim}
Confusion Matrix and Statistics

               
prediction.nnet Benigno Maligno
        Benigno     108       6
        Maligno       7      69
                                          
               Accuracy : 0.9316          
                 95% CI : (0.8858, 0.9631)
    No Information Rate : 0.6053          
    P-Value [Acc > NIR] : <2e-16          
                                          
                  Kappa : 0.8571          
                                          
 Mcnemar's Test P-Value : 1               
                                          
            Sensitivity : 0.9200          
            Specificity : 0.9391          
         Pos Pred Value : 0.9079          
         Neg Pred Value : 0.9474          
             Prevalence : 0.3947          
         Detection Rate : 0.3632          
   Detection Prevalence : 0.4000          
      Balanced Accuracy : 0.9296          
                                          
       'Positive' Class : Maligno         
                                          
\end{verbatim}

\begin{center}\includegraphics{PEC3_ML_files/figure-latex/unnamed-chunk-42-1} \end{center}

El modelo empleando \texttt{caret} con categoria positiva `Maligno'
obtiene una precision de 0.932 y una sensitividad y especificidad de
0.92 y 0.939 respectivamente.

Empleamos ahora \textbf{5-fold crossvalidation} para tratar de mejorar
la clasificación:

\begin{Shaded}
\begin{Highlighting}[]
\CommentTok{# modelo 5-crossvalidation }
\NormalTok{model.nnet}\FloatTok{.2}\NormalTok{ <-}\StringTok{ }\KeywordTok{train}\NormalTok{(diagnosis }\OperatorTok{~}\StringTok{ }\NormalTok{., datos.train, }\DataTypeTok{method=}\StringTok{'nnet'}\NormalTok{, }
               \DataTypeTok{trControl=} \KeywordTok{trainControl}\NormalTok{(}\DataTypeTok{method=}\StringTok{'cv'}\NormalTok{, }\DataTypeTok{number=}\DecValTok{5}\NormalTok{), }
               \DataTypeTok{tuneGrid=} \OtherTok{NULL}\NormalTok{, }\DataTypeTok{tuneLength=}\DecValTok{10}\NormalTok{ ,}\DataTypeTok{trace =} \OtherTok{FALSE}\NormalTok{)}

\KeywordTok{plotnet}\NormalTok{(model.nnet}\FloatTok{.2}\NormalTok{, }\DataTypeTok{alpha=}\FloatTok{0.6}\NormalTok{)}
\CommentTok{#summary(model.nnet.2)}
\NormalTok{prediction.nnet2 <-}\StringTok{ }\KeywordTok{predict}\NormalTok{(model.nnet}\FloatTok{.2}\NormalTok{, datos.test)           }\CommentTok{# predecir los resultados}
\NormalTok{result.caret2 <-}\StringTok{ }\KeywordTok{table}\NormalTok{(prediction.nnet2, datos.test}\OperatorTok{$}\NormalTok{diagnosis)  }\CommentTok{# comparar con los datos test}
\NormalTok{result.caret2}
\end{Highlighting}
\end{Shaded}

\begin{verbatim}
                
prediction.nnet2 Benigno Maligno
         Benigno     107       7
         Maligno       8      68
\end{verbatim}

\begin{Shaded}
\begin{Highlighting}[]
\CommentTok{# la probabilidad para cada clase:}
\NormalTok{prediction.nnet2 <-}\StringTok{ }\KeywordTok{predict}\NormalTok{(model.nnet}\FloatTok{.2}\NormalTok{, datos.test, }\DataTypeTok{type=}\StringTok{"prob"}\NormalTok{)  }
\KeywordTok{head}\NormalTok{(prediction.nnet2)}
\end{Highlighting}
\end{Shaded}

\begin{verbatim}
       Benigno     Maligno
2  0.006467878 0.993532122
4  0.996815731 0.003184269
5  0.006467931 0.993532069
9  0.118938318 0.881061682
11 0.006673474 0.993326526
19 0.006467878 0.993532122
\end{verbatim}

\begin{Shaded}
\begin{Highlighting}[]
\NormalTok{(cmatrix.nnet2 <-}\StringTok{ }\KeywordTok{confusionMatrix}\NormalTok{(result.caret2,}\DataTypeTok{positive =} \StringTok{"Maligno"}\NormalTok{))}
\end{Highlighting}
\end{Shaded}

\begin{verbatim}
Confusion Matrix and Statistics

                
prediction.nnet2 Benigno Maligno
         Benigno     107       7
         Maligno       8      68
                                          
               Accuracy : 0.9211          
                 95% CI : (0.8731, 0.9551)
    No Information Rate : 0.6053          
    P-Value [Acc > NIR] : <2e-16          
                                          
                  Kappa : 0.8352          
                                          
 Mcnemar's Test P-Value : 1               
                                          
            Sensitivity : 0.9067          
            Specificity : 0.9304          
         Pos Pred Value : 0.8947          
         Neg Pred Value : 0.9386          
             Prevalence : 0.3947          
         Detection Rate : 0.3579          
   Detection Prevalence : 0.4000          
      Balanced Accuracy : 0.9186          
                                          
       'Positive' Class : Maligno         
                                          
\end{verbatim}

\begin{center}\includegraphics{PEC3_ML_files/figure-latex/unnamed-chunk-43-1} \end{center}

El modelo empleando \texttt{caret} con 5-fold crossvalidation obtiene
una precision de 0.921 y una sensitividad y especificidad de 0.907 y
0.93 respectivamente.

Como vemos con el paquete \texttt{caret} obtenemos los mejores valores
si no realizamos la 5-fold crossvalidation.

\hypertarget{support-vector-machine}{%
\subsection{Support Vector Machine}\label{support-vector-machine}}

\hypertarget{transformaciuxf3n-de-los-datos-3}{%
\subsubsection{Transformación de los
datos}\label{transformaciuxf3n-de-los-datos-3}}

El SVM no requiere realizar transformaciones normalizantes de los datos.

El porcentaje de muestras tumorales en los datos de training es de
36.15\%, y en los de test de 39.47\%. Así que, como el tamaño de la
muestra es más bien bajo, se podría escoger otros metodos como k-fold
cross validation o bootstrap, que emplearemos después.

\hypertarget{entrenar-el-modelo-3}{%
\subsubsection{Entrenar el modelo}\label{entrenar-el-modelo-3}}

El algoritmo de SVM que se usa es la función \texttt{ksvm()} del paquete
\emph{kernlab}. Se construye el modelo más sencillo, el lineal, usando
como kernel el valor \texttt{vanilladot}:

\begin{Shaded}
\begin{Highlighting}[]
\KeywordTok{library}\NormalTok{(kernlab)}
\KeywordTok{set.seed}\NormalTok{(params}\OperatorTok{$}\NormalTok{seed.clsfier)}
\NormalTok{modeloLineal <-}\StringTok{ }\KeywordTok{ksvm}\NormalTok{(diagnosis}\OperatorTok{~}\NormalTok{.,}\DataTypeTok{data=}\NormalTok{datos.train, }\DataTypeTok{kernel=}\StringTok{'vanilladot'}\NormalTok{)}
\end{Highlighting}
\end{Shaded}

\begin{verbatim}
 Setting default kernel parameters  
\end{verbatim}

\begin{Shaded}
\begin{Highlighting}[]
\CommentTok{# Veamos la información básica sobre el modelo}
\NormalTok{modeloLineal}
\end{Highlighting}
\end{Shaded}

\begin{verbatim}
Support Vector Machine object of class "ksvm" 

SV type: C-svc  (classification) 
 parameter : cost C = 1 

Linear (vanilla) kernel function. 

Number of Support Vectors : 24 

Objective Function Value : -11.728 
Training error : 0.010554 
\end{verbatim}

Se puede observar que la función lineal no tiene parametros adicionales
(`hiperparametros') al de coste.

\hypertarget{predicciuxf3n-y-evaluaciuxf3n-del-algoritmo.}{%
\subsubsection{Predicción y Evaluación del
algoritmo.}\label{predicciuxf3n-y-evaluaciuxf3n-del-algoritmo.}}

Para evaluar el modelo y analizar su rendimiento se realiza la
predicción con nuevos datos: los datos de test.

\begin{Shaded}
\begin{Highlighting}[]
\NormalTok{modLineal.pred <-}\StringTok{ }\KeywordTok{predict}\NormalTok{(modeloLineal, datos.test)}
\end{Highlighting}
\end{Shaded}

Se obtiene la matriz de confusión, usando la función
\texttt{confusionMatrix}, al igual que en los otros algoritmos:

\begin{Shaded}
\begin{Highlighting}[]
\CommentTok{# Modelo lineal}
\NormalTok{res.svm <-}\StringTok{ }\KeywordTok{table}\NormalTok{(modLineal.pred, datos.test}\OperatorTok{$}\NormalTok{diagnosis)}
\NormalTok{(cmatrixSVM <-}\StringTok{ }\KeywordTok{confusionMatrix}\NormalTok{(res.svm, }\DataTypeTok{positive=}\StringTok{"Maligno"}\NormalTok{))}
\end{Highlighting}
\end{Shaded}

\begin{verbatim}
Confusion Matrix and Statistics

              
modLineal.pred Benigno Maligno
       Benigno     112       9
       Maligno       3      66
                                          
               Accuracy : 0.9368          
                 95% CI : (0.8923, 0.9669)
    No Information Rate : 0.6053          
    P-Value [Acc > NIR] : <2e-16          
                                          
                  Kappa : 0.866           
                                          
 Mcnemar's Test P-Value : 0.1489          
                                          
            Sensitivity : 0.8800          
            Specificity : 0.9739          
         Pos Pred Value : 0.9565          
         Neg Pred Value : 0.9256          
             Prevalence : 0.3947          
         Detection Rate : 0.3474          
   Detection Prevalence : 0.3632          
      Balanced Accuracy : 0.9270          
                                          
       'Positive' Class : Maligno         
                                          
\end{verbatim}

El modelo de SVM lineal con categoria positiva `Maligno' obtiene una
precisión de 0.937 y una sensitividad y especificidad de 0.88 y 0.974
respectivamente.

\hypertarget{posible-mejora-kernel-gaussiano}{%
\paragraph{Posible mejora: kernel
Gaussiano}\label{posible-mejora-kernel-gaussiano}}

Ahora se plantea un SVM con un el kernel Gaussiano, \texttt{rbfdot} para
tratar de mejorar el rendimiento:

\begin{Shaded}
\begin{Highlighting}[]
\KeywordTok{set.seed}\NormalTok{(params}\OperatorTok{$}\NormalTok{seed.clsfier)}
\NormalTok{modeloGauss <-}\StringTok{ }\KeywordTok{ksvm}\NormalTok{(diagnosis }\OperatorTok{~}\StringTok{ }\NormalTok{., }\DataTypeTok{data =}\NormalTok{ datos.train, }\DataTypeTok{kernel =} \StringTok{"rbfdot"}\NormalTok{)}
\CommentTok{# Predicción}
\NormalTok{modeloGauss.pred <-}\StringTok{ }\KeywordTok{predict}\NormalTok{(modeloGauss, datos.test)}
\end{Highlighting}
\end{Shaded}

El resultado de la matriz de confusión con los datos de test es:

\begin{Shaded}
\begin{Highlighting}[]
\NormalTok{res.svm.gaus <-}\StringTok{ }\KeywordTok{table}\NormalTok{(modeloGauss.pred, datos.test}\OperatorTok{$}\NormalTok{diagnosis)}
\CommentTok{# Resultados}
\KeywordTok{library}\NormalTok{(caret)}
\NormalTok{(cmatrixSVM2 <-}\StringTok{ }\KeywordTok{confusionMatrix}\NormalTok{(res.svm.gaus, }\DataTypeTok{positive =} \StringTok{"Maligno"}\NormalTok{))}
\end{Highlighting}
\end{Shaded}

\begin{verbatim}
Confusion Matrix and Statistics

                
modeloGauss.pred Benigno Maligno
         Benigno     112       5
         Maligno       3      70
                                          
               Accuracy : 0.9579          
                 95% CI : (0.9187, 0.9816)
    No Information Rate : 0.6053          
    P-Value [Acc > NIR] : <2e-16          
                                          
                  Kappa : 0.9115          
                                          
 Mcnemar's Test P-Value : 0.7237          
                                          
            Sensitivity : 0.9333          
            Specificity : 0.9739          
         Pos Pred Value : 0.9589          
         Neg Pred Value : 0.9573          
             Prevalence : 0.3947          
         Detection Rate : 0.3684          
   Detection Prevalence : 0.3842          
      Balanced Accuracy : 0.9536          
                                          
       'Positive' Class : Maligno         
                                          
\end{verbatim}

El nuevo modelo de SVM con kernel gaussiano obtiene una precision de
0.958 y una sensitividad y especificidad de 0.933 y 0.974
respectivamente. Vemos que el modelo obtenido con SVM gaussiano tiene
una mayor precision.

\hypertarget{paquete-caret}{%
\subsubsection{\texorpdfstring{Paquete
\emph{caret}}{Paquete caret}}\label{paquete-caret}}

Se construye un nuevo modelo con el objetivo de comprobar si la
implementación del paquete \texttt{caret} produce diferencias. Se evalúa
el rendimiento con la partición train/test y 5-fold cross validation en
el modelo lineal y bootstrap con el modelo radial.

Para crear la partición de los datos en training/test tenemos la
posibilidad de emplear la función \texttt{createDataPartition} del
paquete \texttt{caret}. Como curiosidad veamos que ocurre con los
resultados, pero posteriormente emplearemos el mismo set de datos de
training/test creados en el apartado 3.3, para poder comprar los
resultados.

\begin{Shaded}
\begin{Highlighting}[]
\KeywordTok{library}\NormalTok{(caret)}
\CommentTok{# Partición de datos}
\KeywordTok{set.seed}\NormalTok{(params}\OperatorTok{$}\NormalTok{seed.train)}
\CommentTok{# We wish 75% for the trainset }
\NormalTok{inTrain <-}\StringTok{ }\KeywordTok{createDataPartition}\NormalTok{(}\DataTypeTok{y=}\NormalTok{data}\OperatorTok{$}\NormalTok{diagnosis, }\DataTypeTok{p=}\NormalTok{params}\OperatorTok{$}\NormalTok{p.train, }\DataTypeTok{list=}\OtherTok{FALSE}\NormalTok{)}

\NormalTok{train.set <-}\StringTok{ }\NormalTok{data[inTrain,]}
\NormalTok{test.set  <-}\StringTok{ }\NormalTok{data[}\OperatorTok{-}\NormalTok{inTrain,]}

\KeywordTok{nrow}\NormalTok{(train.set)}\OperatorTok{/}\KeywordTok{nrow}\NormalTok{(test.set) }\CommentTok{# debe estar alrededor de 2}
\end{Highlighting}
\end{Shaded}

\begin{verbatim}
[1] 2.010582
\end{verbatim}

Entrenamos el modelo con los datos de train y test creados por el
paquete \texttt{caret}:

\begin{Shaded}
\begin{Highlighting}[]
\CommentTok{# modelo sólo de Train sin repetición}
\KeywordTok{set.seed}\NormalTok{(params}\OperatorTok{$}\NormalTok{seed.otro)}
\NormalTok{model.svm.caret <-}\StringTok{ }\KeywordTok{train}\NormalTok{(diagnosis }\OperatorTok{~}\StringTok{ }\NormalTok{., train.set, }\DataTypeTok{method=}\StringTok{'svmLinear'}\NormalTok{, }
               \DataTypeTok{trControl=} \KeywordTok{trainControl}\NormalTok{(}\DataTypeTok{method=}\StringTok{'none'}\NormalTok{), }
                \DataTypeTok{tuneGrid=} \OtherTok{NULL}\NormalTok{, }\DataTypeTok{trace =} \OtherTok{FALSE}\NormalTok{) }\CommentTok{#}

\NormalTok{prediction.svm.caret <-}\StringTok{ }\KeywordTok{predict}\NormalTok{(model.svm.caret, test.set)         }\CommentTok{# predecir los resultados}
\NormalTok{res.svm.caret <-}\StringTok{ }\KeywordTok{table}\NormalTok{(prediction.svm.caret, test.set}\OperatorTok{$}\NormalTok{diagnosis)   }\CommentTok{# comparar con los datos test}

\KeywordTok{confusionMatrix}\NormalTok{(res.svm.caret, }\DataTypeTok{positive=}\StringTok{"Maligno"}\NormalTok{)   }
\end{Highlighting}
\end{Shaded}

\begin{verbatim}
Confusion Matrix and Statistics

                    
prediction.svm.caret Benigno Maligno
             Benigno     113       2
             Maligno       6      68
                                          
               Accuracy : 0.9577          
                 95% CI : (0.9183, 0.9816)
    No Information Rate : 0.6296          
    P-Value [Acc > NIR] : <2e-16          
                                          
                  Kappa : 0.9103          
                                          
 Mcnemar's Test P-Value : 0.2888          
                                          
            Sensitivity : 0.9714          
            Specificity : 0.9496          
         Pos Pred Value : 0.9189          
         Neg Pred Value : 0.9826          
             Prevalence : 0.3704          
         Detection Rate : 0.3598          
   Detection Prevalence : 0.3915          
      Balanced Accuracy : 0.9605          
                                          
       'Positive' Class : Maligno         
                                          
\end{verbatim}

Ahora se repite el modelo pero con los mismos datos de training/test
usados en la función \texttt{ksvm} con kernel \texttt{vanilladot} del
paquete \emph{kernlab}, para comprobar las variaciones:

\begin{Shaded}
\begin{Highlighting}[]
\CommentTok{# modelo sin repetición}
\KeywordTok{set.seed}\NormalTok{(params}\OperatorTok{$}\NormalTok{seed.otro)}
\NormalTok{model.svm.caret2 <-}\StringTok{ }\KeywordTok{train}\NormalTok{(diagnosis }\OperatorTok{~}\StringTok{ }\NormalTok{., datos.train, }\DataTypeTok{method=}\StringTok{'svmLinear'}\NormalTok{, }
               \DataTypeTok{trControl=} \KeywordTok{trainControl}\NormalTok{(}\DataTypeTok{method=}\StringTok{'none'}\NormalTok{), }
                \DataTypeTok{tuneGrid=} \OtherTok{NULL}\NormalTok{, }\DataTypeTok{trace =} \OtherTok{FALSE}\NormalTok{)}

\NormalTok{prediction.svm.caret2 <-}\StringTok{ }\KeywordTok{predict}\NormalTok{(model.svm.caret2, datos.test)          }\CommentTok{# predecir los resultados}
\NormalTok{res.svm.caret2 <-}\StringTok{ }\KeywordTok{table}\NormalTok{(prediction.svm.caret2, datos.test}\OperatorTok{$}\NormalTok{diagnosis)    }\CommentTok{# comparar con los datos test}

\NormalTok{svm.caret.matrix <-}\StringTok{ }\KeywordTok{confusionMatrix}\NormalTok{(res.svm.caret2, }\DataTypeTok{positive=}\StringTok{"Maligno"}\NormalTok{)   }
\end{Highlighting}
\end{Shaded}

El modelo de SVM lineal con categoria positiva `Maligno' obtiene una
precisión de 0.937 y una sensitividad y especificidad de 0.88 y 0.974
respectivamente.

Como vemos los datos empleados en el análisis son importantes ya que
pueden cambiar el resultado de los datos. En este caso la selección de
datos empleada por el paquete \texttt{caret} mejora tanto la precisión
como la sensibilidad y la especificidad.

Empleamos ahora \textbf{5-fold crossvalidation} para tratar de mejorar
la clasificación:

\begin{Shaded}
\begin{Highlighting}[]
\CommentTok{# modelo 5-crossvalidation }
\KeywordTok{set.seed}\NormalTok{(params}\OperatorTok{$}\NormalTok{seed.clsfier)}
\NormalTok{model.svm.fold <-}\StringTok{ }\KeywordTok{train}\NormalTok{(diagnosis }\OperatorTok{~}\StringTok{ }\NormalTok{., datos.train, }\DataTypeTok{method=}\StringTok{'svmLinear'}\NormalTok{, }
               \DataTypeTok{trControl=} \KeywordTok{trainControl}\NormalTok{(}\DataTypeTok{method=}\StringTok{'cv'}\NormalTok{, }\DataTypeTok{number=}\DecValTok{5}\NormalTok{), }
                \DataTypeTok{tuneGrid=} \OtherTok{NULL}\NormalTok{, }\DataTypeTok{trace =} \OtherTok{FALSE}\NormalTok{)}

\NormalTok{prediction.fold <-}\StringTok{ }\KeywordTok{predict}\NormalTok{(model.svm.fold, datos.test)      }\CommentTok{# predecir los resultados}
\NormalTok{res.fold <-}\StringTok{ }\KeywordTok{table}\NormalTok{(prediction.fold, datos.test}\OperatorTok{$}\NormalTok{diagnosis)    }\CommentTok{# comparar con los datos test}

\NormalTok{svm.caret.matrix5 <-}\StringTok{ }\KeywordTok{confusionMatrix}\NormalTok{(res.fold, }\DataTypeTok{positive=}\StringTok{"Maligno"}\NormalTok{)   }
\end{Highlighting}
\end{Shaded}

El modelo de SVM lineal con categoria positiva `Maligno' obtiene una
precisión de 0.937 y una sensitividad y especificidad de 0.88 y 0.974
respectivamente.

Obtenemos datos similares empleando el paquete \texttt{caret} pero sin
5-fold crossvalidation.

Por último empleamos bootstrap con el modelo SVM \texttt{svmRadial}:

\begin{Shaded}
\begin{Highlighting}[]
\CommentTok{# Por defecto es Bootstrap, con 25 repeticiones para 3 posibles decay}
\CommentTok{# y 3 posibles sizes}
\KeywordTok{set.seed}\NormalTok{(params}\OperatorTok{$}\NormalTok{seed.clsfier)}
\NormalTok{model.svm.caret3 <-}\StringTok{ }\KeywordTok{train}\NormalTok{(diagnosis }\OperatorTok{~}\StringTok{ }\NormalTok{., datos.train, }\DataTypeTok{method=}\StringTok{'svmRadial'}\NormalTok{, }\DataTypeTok{trace =} \OtherTok{FALSE}\NormalTok{)}

\NormalTok{prediction.boots <-}\StringTok{ }\KeywordTok{predict}\NormalTok{(model.svm.caret3, datos.test)      }\CommentTok{# predecir los resultados}
\NormalTok{res.svm.boots <-}\StringTok{ }\KeywordTok{table}\NormalTok{(prediction.boots, datos.test}\OperatorTok{$}\NormalTok{diagnosis) }\CommentTok{# comparar con los datos test}

\NormalTok{res.svm.boots.matrix <-}\StringTok{ }\KeywordTok{confusionMatrix}\NormalTok{(res.svm.boots, }\DataTypeTok{positive=}\StringTok{"Maligno"}\NormalTok{)}
\NormalTok{res.svm.boots.matrix}
\end{Highlighting}
\end{Shaded}

\begin{verbatim}
Confusion Matrix and Statistics

                
prediction.boots Benigno Maligno
         Benigno     112       5
         Maligno       3      70
                                          
               Accuracy : 0.9579          
                 95% CI : (0.9187, 0.9816)
    No Information Rate : 0.6053          
    P-Value [Acc > NIR] : <2e-16          
                                          
                  Kappa : 0.9115          
                                          
 Mcnemar's Test P-Value : 0.7237          
                                          
            Sensitivity : 0.9333          
            Specificity : 0.9739          
         Pos Pred Value : 0.9589          
         Neg Pred Value : 0.9573          
             Prevalence : 0.3947          
         Detection Rate : 0.3684          
   Detection Prevalence : 0.3842          
      Balanced Accuracy : 0.9536          
                                          
       'Positive' Class : Maligno         
                                          
\end{verbatim}

El modelo de SVM radial con validación bootstrap obtiene una precisión
de 0.958 y una sensitividad y especificidad de 0.933 y 0.974
respectivamente.

\hypertarget{uxe1rbol-de-decisiuxf3n}{%
\subsection{Árbol de Decisión}\label{uxe1rbol-de-decisiuxf3n}}

\hypertarget{transformaciuxf3n-de-los-datos-4}{%
\subsubsection{Transformación de los
datos}\label{transformaciuxf3n-de-los-datos-4}}

El algorítmo de árboles de decisión no requiere realizar
transformaciones normalizantes de los datos, empleamos los datos de
entrenamiento y test creados en el partado 3.3.

\hypertarget{entrenar-el-modelo-4}{%
\subsubsection{Entrenar el modelo}\label{entrenar-el-modelo-4}}

\begin{Shaded}
\begin{Highlighting}[]
\CommentTok{# fijamos la semilla para el clasificador}
\KeywordTok{library}\NormalTok{(C50)}
\KeywordTok{set.seed}\NormalTok{(params}\OperatorTok{$}\NormalTok{seed.clsfier)}
\NormalTok{data.model.ab<-}\StringTok{ }\KeywordTok{C5.0}\NormalTok{(datos.train[}\OperatorTok{-}\DecValTok{31}\NormalTok{], datos.train}\OperatorTok{$}\NormalTok{diagnosis)}
\NormalTok{data.model.ab}
\end{Highlighting}
\end{Shaded}

\begin{verbatim}

Call:
C5.0.default(x = datos.train[-31], y = datos.train$diagnosis)

Classification Tree
Number of samples: 379 
Number of predictors: 30 

Tree size: 8 

Non-standard options: attempt to group attributes
\end{verbatim}

Para ver todos los detalles del árbol de decisiones creado podemos usar
\textbf{summary}:

\begin{Shaded}
\begin{Highlighting}[]
\NormalTok{res.ab<-}\KeywordTok{summary}\NormalTok{(data.model.ab)}
\NormalTok{res.ab}
\end{Highlighting}
\end{Shaded}

\begin{verbatim}

Call:
C5.0.default(x = datos.train[-31], y = datos.train$diagnosis)


C5.0 [Release 2.07 GPL Edition]     Tue Jun 09 14:01:47 2020
-------------------------------

Class specified by attribute `outcome'

Read 379 cases (31 attributes) from undefined.data

Decision tree:

perimeter_worst > 115.9: Maligno (111)
perimeter_worst <= 115.9:
:...concave.points_worst > 0.1466:
    :...texture_worst <= 23.73: Benigno (6/1)
    :   texture_worst > 23.73: Maligno (18/1)
    concave.points_worst <= 0.1466:
    :...perimeter_worst <= 102.3: Benigno (212/1)
        perimeter_worst > 102.3:
        :...texture_mean <= 21.01: Benigno (21/1)
            texture_mean > 21.01:
            :...radius_worst > 16.67: Maligno (4)
                radius_worst <= 16.67:
                :...texture_mean <= 22.61: Maligno (2)
                    texture_mean > 22.61: Benigno (5)


Evaluation on training data (379 cases):

        Decision Tree   
      ----------------  
      Size      Errors  

         8    4( 1.1%)   <<


       (a)   (b)    <-classified as
      ----  ----
       241     1    (a): class Benigno
         3   134    (b): class Maligno


    Attribute usage:

    100.00% perimeter_worst
     70.71% concave.points_worst
      8.44% texture_mean
      6.33% texture_worst
      2.90% radius_worst


Time: 0.0 secs
\end{verbatim}

El resultado anterior muestra las ramas en el árbol de decisión. Después
del árbol, el resultado del resumen muestra una \textbf{matriz de
confusión}, que nos indica los registros incorrectamente clasificados
por el modelo. Se observa que este modelo clasifica adecuadamente el
diagnóstico de todas las muestras excepto 1 de las 379 muestras, con una
tasa de error del 1,1\%. Un total de 134 valores reales no se
clasificaron incorrectamente como (falsos positivos), mientras que 3
valores se clasificaron erróneamente como Malignos (falsos negativos).

\begin{Shaded}
\begin{Highlighting}[]
\KeywordTok{plot}\NormalTok{(data.model.ab, }\DataTypeTok{subtree=}\DecValTok{2}\NormalTok{) }
\end{Highlighting}
\end{Shaded}

\begin{center}\includegraphics{PEC3_ML_files/figure-latex/unnamed-chunk-56-1} \end{center}

\hypertarget{predicciuxf3n-y-evaluaciuxf3n-del-algoritmo-3}{%
\subsubsection{Predicción y Evaluación del
algoritmo}\label{predicciuxf3n-y-evaluaciuxf3n-del-algoritmo-3}}

La salida anterior debemos validarla con los datos de test:

\begin{Shaded}
\begin{Highlighting}[]
\NormalTok{data.predict.ab<-}\StringTok{ }\KeywordTok{predict}\NormalTok{(data.model.ab, datos.test)}
\end{Highlighting}
\end{Shaded}

Creamos la tabla de resultados con la función \texttt{confusionMatrix()}
de \texttt{caret}:

\begin{Shaded}
\begin{Highlighting}[]
\NormalTok{cf.arbol<-}\KeywordTok{confusionMatrix}\NormalTok{(data.predict.ab, datos.test}\OperatorTok{$}\NormalTok{diagnosis, }\DataTypeTok{positive =} \StringTok{"Maligno"}\NormalTok{)}
\NormalTok{cf.arbol}
\end{Highlighting}
\end{Shaded}

\begin{verbatim}
Confusion Matrix and Statistics

          Reference
Prediction Benigno Maligno
   Benigno     110       8
   Maligno       5      67
                                          
               Accuracy : 0.9316          
                 95% CI : (0.8858, 0.9631)
    No Information Rate : 0.6053          
    P-Value [Acc > NIR] : <2e-16          
                                          
                  Kappa : 0.8558          
                                          
 Mcnemar's Test P-Value : 0.5791          
                                          
            Sensitivity : 0.8933          
            Specificity : 0.9565          
         Pos Pred Value : 0.9306          
         Neg Pred Value : 0.9322          
             Prevalence : 0.3947          
         Detection Rate : 0.3526          
   Detection Prevalence : 0.3789          
      Balanced Accuracy : 0.9249          
                                          
       'Positive' Class : Maligno         
                                          
\end{verbatim}

De las 0 muestras incluidas en los datos de test, nuestro modelo predijo
correctamente que 110 como \textbf{Maligno} y 67 como \textbf{Benigno},
lo que resulta una precisión del \% y una tasa de error del \%.

Este resultado es peor que su rendimiento en los datos de entrenamiento,
pero no es inesperado, dado que el rendimiento de un modelo a menudo es
peor en datos no vistos.

Si nos fijamos en \textbf{Sensitivity}, sensibilidad del modelo, vemos
que es 0.89, y si nos fijamos en \textbf{Specificity}, ratio de
verdaderos negativos, que nos mide la proporción de negativos que han
sido correctamente clasificados, vemos que es 0.96.

En resumen:

\begin{table}

\caption{\label{tab:unnamed-chunk-59}ANÁLISIS USANDO ÁRBOLES DE CLASIFICACIÓN}
\centering
\begin{tabular}[t]{l|c|c|c}
\hline
modelo & Accuracy & Kappa & Sensitivity\\
\hline
C5.0 & 0.932 & 0.856 & 0.89\\
\hline
\end{tabular}
\end{table}

\hypertarget{posible-mejora-algoritmo-c-4.5}{%
\paragraph{Posible mejora: algoritmo C
4.5}\label{posible-mejora-algoritmo-c-4.5}}

Comprobamos si empleando el algoritmo C 4.5, mediante la adición de
refuerzo adaptativo, mejora el modelo. Este es un proceso en el que se
construyen muchos árboles de decisión y los árboles votan en la mejor
clase para cada ejemplo.

La función \texttt{C5.0()} facilita agregar aumentos a nuestro árbol de
decisión C5.0. Simplemente necesitamos agregar un parámetro adicional
\textbf{trials} que indique el \textbf{número de árboles de decisión
separados} para usar en el modelo. Este parámetro establece un límite
superior; el algoritmo dejará de agregar árboles si reconoce que las
pruebas adicionales no parecen mejorar la precisión. Comenzaremos con
\texttt{trials=10} , un número qestándar, ya que las investigaciones
sugieren que esto reduce las tasas de error en los datos de prueba en
aproximadamente un 25 por ciento:

\begin{Shaded}
\begin{Highlighting}[]
\CommentTok{# fijamos la semilla para el clasificador}
\KeywordTok{set.seed}\NormalTok{(params}\OperatorTok{$}\NormalTok{seed.clsfier)}
\NormalTok{model.boost10 <-}\StringTok{ }\KeywordTok{C5.0}\NormalTok{(datos.train[}\OperatorTok{-}\DecValTok{31}\NormalTok{], datos.train}\OperatorTok{$}\NormalTok{diagnosis, }\DataTypeTok{trials =} \DecValTok{10}\NormalTok{)}
\KeywordTok{summary}\NormalTok{(model.boost10)}
\end{Highlighting}
\end{Shaded}

\begin{verbatim}

Call:
C5.0.default(x = datos.train[-31], y = datos.train$diagnosis, trials = 10)


C5.0 [Release 2.07 GPL Edition]     Tue Jun 09 14:01:48 2020
-------------------------------

Class specified by attribute `outcome'

Read 379 cases (31 attributes) from undefined.data

-----  Trial 0:  -----

Decision tree:

perimeter_worst > 115.9: Maligno (111)
perimeter_worst <= 115.9:
:...concave.points_worst > 0.1466:
    :...texture_worst <= 23.73: Benigno (6/1)
    :   texture_worst > 23.73: Maligno (18/1)
    concave.points_worst <= 0.1466:
    :...perimeter_worst <= 102.3: Benigno (212/1)
        perimeter_worst > 102.3:
        :...texture_mean <= 21.01: Benigno (21/1)
            texture_mean > 21.01:
            :...radius_worst > 16.67: Maligno (4)
                radius_worst <= 16.67:
                :...texture_mean <= 22.61: Maligno (2)
                    texture_mean > 22.61: Benigno (5)

-----  Trial 1:  -----

Decision tree:

area_worst <= 645.8: Benigno (161.2/0.8)
area_worst > 645.8:
:...concavity_worst <= 0.206: Benigno (28.6/0.8)
    concavity_worst > 0.206:
    :...texture_mean <= 15.51: Benigno (9.8/1.5)
        texture_mean > 15.51: Maligno (179.4/9)

-----  Trial 2:  -----

Decision tree:

concave.points_worst > 0.1599: Maligno (92.6/0.6)
concave.points_worst <= 0.1599:
:...radius_worst <= 16.51: Benigno (230.3/21.8)
    radius_worst > 16.51: Maligno (56.1/15.4)

-----  Trial 3:  -----

Decision tree:

concave.points_worst > 0.1708: Maligno (66.2)
concave.points_worst <= 0.1708:
:...texture_worst <= 29.51: Benigno (223/28.8)
    texture_worst > 29.51:
    :...radius_worst <= 14.42: Benigno (12)
        radius_worst > 14.42:
        :...compactness_se > 0.0431: Benigno (4.9)
            compactness_se <= 0.0431:
            :...smoothness_worst <= 0.1086: Benigno (6.2)
                smoothness_worst > 0.1086: Maligno (66.6/1.3)

-----  Trial 4:  -----

Decision tree:

concave.points_mean > 0.04908:
:...area_se <= 18.04: Benigno (18.5)
:   area_se > 18.04: Maligno (173.1/7.5)
concave.points_mean <= 0.04908:
:...symmetry_worst <= 0.2827: Benigno (105.4/0.7)
    symmetry_worst > 0.2827:
    :...fractal_dimension_mean <= 0.06065: Maligno (41.3/7.5)
        fractal_dimension_mean > 0.06065: Benigno (40.7/0.3)

-----  Trial 5:  -----

Decision tree:

perimeter_worst > 115.9: Maligno (79.7)
perimeter_worst <= 115.9:
:...smoothness_worst <= 0.1408: Benigno (216/24.8)
    smoothness_worst > 0.1408:
    :...area_worst <= 645.8: Benigno (21.5)
        area_worst > 645.8: Maligno (61.8/6.4)

-----  Trial 6:  -----

Decision tree:

area_se <= 18.51: Benigno (74.4)
area_se > 18.51:
:...concavity_worst <= 0.1546: Benigno (33.5)
    concavity_worst > 0.1546:
    :...texture_worst > 25.82:
        :...symmetry_mean <= 0.1516: Benigno (12/2.9)
        :   symmetry_mean > 0.1516: Maligno (150.5/6.7)
        texture_worst <= 25.82:
        :...perimeter_worst > 116.6: Maligno (23)
            perimeter_worst <= 116.6:
            :...fractal_dimension_worst <= 0.09782: Benigno (59)
                fractal_dimension_worst > 0.09782: Maligno (26.6/9.6)

-----  Trial 7:  -----

Decision tree:

area_worst > 862.1:
:...radius_se <= 0.2387: Benigno (6/0.2)
:   radius_se > 0.2387: Maligno (94.5)
area_worst <= 862.1:
:...perimeter_worst <= 91.88: Benigno (106.2)
    perimeter_worst > 91.88:
    :...perimeter_worst <= 92.04: Maligno (16.1/0.3)
        perimeter_worst > 92.04:
        :...concave.points_worst <= 0.1108: Benigno (48.8)
            concave.points_worst > 0.1108:
            :...compactness_se > 0.03889: Benigno (32/0.5)
                compactness_se <= 0.03889:
                :...texture_worst <= 20.35: Benigno (16.7)
                    texture_worst > 20.35: Maligno (58.7/9.4)

-----  Trial 8:  -----

Decision tree:

perimeter_worst > 115.9: Maligno (65.3)
perimeter_worst <= 115.9:
:...concave.points_worst > 0.1712: Maligno (31)
    concave.points_worst <= 0.1712:
    :...texture_mean <= 21.57: Benigno (201.2/11.1)
        texture_mean > 21.57:
        :...area_worst <= 639.1: Benigno (21.2)
            area_worst > 639.1: Maligno (60.2/13.7)

-----  Trial 9:  -----

Decision tree:

concave.points_mean <= 0.04908:
:...radius_worst <= 16.82: Benigno (206.4/9.4)
:   radius_worst > 16.82: Maligno (16.1/3.4)
concave.points_mean > 0.04908:
:...area_se <= 18.04: Benigno (15)
    area_se > 18.04:
    :...compactness_se > 0.07217: Benigno (9.1/0.3)
        compactness_se <= 0.07217:
        :...texture_worst <= 20.35: Benigno (8.6/2.1)
            texture_worst > 20.35: Maligno (123.9/1.8)


Evaluation on training data (379 cases):

Trial       Decision Tree   
-----     ----------------  
      Size      Errors  

   0         8    4( 1.1%)
   1         4   16( 4.2%)
   2         3   15( 4.0%)
   3         6   25( 6.6%)
   4         5   37( 9.8%)
   5         4   11( 2.9%)
   6         7   25( 6.6%)
   7         8   11( 2.9%)
   8         5   16( 4.2%)
   9         6   14( 3.7%)
boost             0( 0.0%)   <<


       (a)   (b)    <-classified as
      ----  ----
       242          (a): class Benigno
             137    (b): class Maligno


    Attribute usage:

    100.00% concave.points_mean
    100.00% area_se
    100.00% texture_worst
    100.00% perimeter_worst
    100.00% area_worst
    100.00% concave.points_worst
     99.74% texture_mean
     75.46% radius_worst
     75.20% smoothness_worst
     74.93% concavity_worst
     61.48% symmetry_worst
     44.06% compactness_se
     33.25% symmetry_mean
     31.66% radius_se
     21.64% fractal_dimension_mean
     11.08% fractal_dimension_worst


Time: 0.0 secs
\end{verbatim}

El clasificador cometió un error de 4 en 379 registros de datos de
entrenamiento para una tasa de error de 1.06 por ciento. Representa una
gran mejora con respecto a la tasa de error del modelo anterior antes de
agregar potenciación. Veamos queda por ver si hay una mejora en los
datos de prueba:

\begin{Shaded}
\begin{Highlighting}[]
\NormalTok{model.boost.pred10 <-}\StringTok{ }\KeywordTok{predict}\NormalTok{(model.boost10, datos.test)}

\NormalTok{cf.arbol}\FloatTok{.10}\NormalTok{<-}\KeywordTok{confusionMatrix}\NormalTok{(model.boost.pred10, datos.test}\OperatorTok{$}\NormalTok{diagnosis, }\DataTypeTok{positive =} \StringTok{"Maligno"}\NormalTok{)}
\NormalTok{cf.arbol}\FloatTok{.10}
\end{Highlighting}
\end{Shaded}

\begin{verbatim}
Confusion Matrix and Statistics

          Reference
Prediction Benigno Maligno
   Benigno     109       7
   Maligno       6      68
                                          
               Accuracy : 0.9316          
                 95% CI : (0.8858, 0.9631)
    No Information Rate : 0.6053          
    P-Value [Acc > NIR] : <2e-16          
                                          
                  Kappa : 0.8565          
                                          
 Mcnemar's Test P-Value : 1               
                                          
            Sensitivity : 0.9067          
            Specificity : 0.9478          
         Pos Pred Value : 0.9189          
         Neg Pred Value : 0.9397          
             Prevalence : 0.3947          
         Detection Rate : 0.3579          
   Detection Prevalence : 0.3895          
      Balanced Accuracy : 0.9272          
                                          
       'Positive' Class : Maligno         
                                          
\end{verbatim}

La precisión se ha mantenido respecto al modelo C5.0 sin boosting. Sin
embargo la sensibilidad mejora levemente.

Accuracy es de 0.932\\
Kappa es de 0.856

\hypertarget{paquete-caret-1}{%
\subsubsection{\texorpdfstring{Paquete
\emph{caret}}{Paquete caret}}\label{paquete-caret-1}}

Empleamos los mismos grupos de train y test que hemos empleado en el
análisis anterior. Para ajustar el modelo se emplea la función
\texttt{train} y empleamos 5-fold crossvalidation para entrenarlo:

\begin{Shaded}
\begin{Highlighting}[]
\CommentTok{## método : cv K-fold cross validation}
\CommentTok{## número : K folds}
\NormalTok{ctrl <-}\StringTok{ }\KeywordTok{trainControl}\NormalTok{( }\DataTypeTok{method=}\StringTok{"cv"}\NormalTok{,}
                      \DataTypeTok{number=}\DecValTok{5}\NormalTok{,}
                      \DataTypeTok{selectionFunction=} \StringTok{"oneSE"}\NormalTok{) }

\CommentTok{## Parámetros de Grid para algortimo de C50}
\CommentTok{## trials -> número de iteraciones boosting }
\NormalTok{grid_C50 <-}\StringTok{ }\KeywordTok{expand.grid}\NormalTok{(}\DataTypeTok{model=}\StringTok{"tree"}\NormalTok{, }\DataTypeTok{trials=}\KeywordTok{c}\NormalTok{(}\DecValTok{5}\NormalTok{,}\DecValTok{10}\NormalTok{,}\DecValTok{20}\NormalTok{,}\DecValTok{30}\NormalTok{),}\DataTypeTok{winnow=}\StringTok{"FALSE"}\NormalTok{)}

\CommentTok{# fijamos la semilla para el clasificador}
\KeywordTok{set.seed}\NormalTok{(params}\OperatorTok{$}\NormalTok{seed.clsfier)}
\CommentTok{## trace <- FALSE para suprimir las iteraciones de entrenamiento}
\NormalTok{modeloC5}\FloatTok{.0}\NormalTok{     <-}\StringTok{ }\KeywordTok{train}\NormalTok{ (diagnosis }\OperatorTok{~}\StringTok{ }\NormalTok{.,}
                  \DataTypeTok{data =}\NormalTok{ datos.train,}
                  \DataTypeTok{method =}\StringTok{"C5.0"}\NormalTok{,}
                  \DataTypeTok{trControl=}\NormalTok{ctrl,}
                  \DataTypeTok{tuneGrid =}\NormalTok{ grid_C50, }
                  \DataTypeTok{metric=}\StringTok{"Accuracy"}\NormalTok{,}
                  \DataTypeTok{prePoc =} \KeywordTok{c}\NormalTok{(}\StringTok{"center"}\NormalTok{, }\StringTok{"scale"}\NormalTok{),}
                  \DataTypeTok{verbose =}\OtherTok{FALSE}\NormalTok{,}
                  \DataTypeTok{trace =} \OtherTok{FALSE}\NormalTok{)}
\NormalTok{modeloC5}\FloatTok{.0}
\end{Highlighting}
\end{Shaded}

\begin{verbatim}
C5.0 

379 samples
 30 predictor
  2 classes: 'Benigno', 'Maligno' 

No pre-processing
Resampling: Cross-Validated (5 fold) 
Summary of sample sizes: 302, 303, 304, 304, 303 
Resampling results across tuning parameters:

  trials  Accuracy   Kappa    
   5      0.9683140  0.9305421
  10      0.9709114  0.9362221
  20      0.9710148  0.9370581
  30      0.9683140  0.9309235

Tuning parameter 'model' was held constant at a value of tree
Tuning parameter
 'winnow' was held constant at a value of FALSE
Accuracy was used to select the optimal model using  the one SE rule.
The final values used for the model were trials = 5, model = tree and winnow = FALSE.
\end{verbatim}

para la evaliuación del rendimiento creamos la matriz de resultados:

\begin{Shaded}
\begin{Highlighting}[]
\CommentTok{# Predicción de Clases}
\NormalTok{prd.c50 <-}\StringTok{ }\KeywordTok{predict}\NormalTok{ ( modeloC5}\FloatTok{.0}\NormalTok{, }\DataTypeTok{newdata =}\NormalTok{ datos.test)}
\CommentTok{# Matriz de Confusión}
\NormalTok{(cf.c50 <-}\StringTok{ }\KeywordTok{confusionMatrix}\NormalTok{( }\DataTypeTok{data=}\NormalTok{prd.c50, datos.test}\OperatorTok{$}\NormalTok{diagnosis, }\DataTypeTok{positive=}\StringTok{"Maligno"}\NormalTok{))}
\end{Highlighting}
\end{Shaded}

\begin{verbatim}
Confusion Matrix and Statistics

          Reference
Prediction Benigno Maligno
   Benigno     113       7
   Maligno       2      68
                                         
               Accuracy : 0.9526         
                 95% CI : (0.912, 0.9781)
    No Information Rate : 0.6053         
    P-Value [Acc > NIR] : <2e-16         
                                         
                  Kappa : 0.8997         
                                         
 Mcnemar's Test P-Value : 0.1824         
                                         
            Sensitivity : 0.9067         
            Specificity : 0.9826         
         Pos Pred Value : 0.9714         
         Neg Pred Value : 0.9417         
             Prevalence : 0.3947         
         Detection Rate : 0.3579         
   Detection Prevalence : 0.3684         
      Balanced Accuracy : 0.9446         
                                         
       'Positive' Class : Maligno        
                                         
\end{verbatim}

La precisión ha aumentado respecto al modelo anterior sin
\texttt{caret}.

Accuracy es de 0.953\\
Kappa es de 0.9.

\hypertarget{random-forest}{%
\subsection{Random Forest}\label{random-forest}}

\hypertarget{transformaciuxf3n-de-los-datos-5}{%
\subsubsection{Transformación de los
datos}\label{transformaciuxf3n-de-los-datos-5}}

El algorítmo de \emph{Random Forest} no requiere realizar
transformaciones normalizantes de los datos, empleamos los datos de
entrenamiento y test creados en el partado 3.3 y empleados en la mayoria
de los algoritmos.

\hypertarget{entrenar-el-modelo-5}{%
\subsubsection{Entrenar el modelo}\label{entrenar-el-modelo-5}}

La función \texttt{randomForest()} crea por defecto un conjunto de 500
árboles, donde cada uno de ellos elige \(\sqrt(p)\) variables de forma
aleatoria para cada árbol, donde \texttt{p} es el número de variables en
el conjunto de datos de entrenamiento.

El objetivo de utilizar una gran cantidad de árboles es entrenar lo
suficiente para que cada variable tenga la oportunidad de aparecer en
varios modelos, base del parámetro \texttt{mtry}. El uso de este valor
limita las variables lo suficiente como para que ocurra una variación
aleatoria sustancial de árbol a árbol.

Empleamos como en los casos antiores la variable factor ``diagnosis''
para que el método ejecutado sea \textbf{classification}, de lo
contrario se ejecuta \textbf{regresion}:

\begin{Shaded}
\begin{Highlighting}[]
\KeywordTok{library}\NormalTok{(randomForest)}
\KeywordTok{set.seed}\NormalTok{(params}\OperatorTok{$}\NormalTok{seed.clsfier)}
\NormalTok{randomf <-}\StringTok{ }\KeywordTok{randomForest}\NormalTok{(diagnosis }\OperatorTok{~}\StringTok{ }\NormalTok{., }\DataTypeTok{data  =}\NormalTok{ datos.train)}
\NormalTok{randomf}
\end{Highlighting}
\end{Shaded}

\begin{verbatim}

Call:
 randomForest(formula = diagnosis ~ ., data = datos.train) 
               Type of random forest: classification
                     Number of trees: 500
No. of variables tried at each split: 5

        OOB estimate of  error rate: 2.9%
Confusion matrix:
        Benigno Maligno class.error
Benigno     238       4  0.01652893
Maligno       7     130  0.05109489
\end{verbatim}

El resultado indica que \emph{Random Forest} incluyó 500 árboles y probó
5 variables en cada división. Hagamos un gráfico del modelo:

\begin{Shaded}
\begin{Highlighting}[]
\KeywordTok{plot}\NormalTok{(randomf, }\DataTypeTok{main=}\StringTok{"modelo Random Forest"}\NormalTok{)}
\end{Highlighting}
\end{Shaded}

\begin{center}\includegraphics{PEC3_ML_files/figure-latex/unnamed-chunk-65-1} \end{center}

Donde la línea roja representa el error cometido al intentar predecir el
diagnóstico, y la línea verde es el error en la predicción
\textbf{Benigno}. Podemos ver la mmportancia de cada una de las
variables en la clasificación del modelo:

\begin{Shaded}
\begin{Highlighting}[]
\KeywordTok{head}\NormalTok{(}\KeywordTok{importance}\NormalTok{(randomf))}
\end{Highlighting}
\end{Shaded}

\begin{verbatim}
                 MeanDecreaseGini
radius_mean             6.1171750
texture_mean            2.2982140
perimeter_mean          8.6155845
area_mean               4.4807408
smoothness_mean         0.7773583
compactness_mean        1.4580878
\end{verbatim}

\begin{Shaded}
\begin{Highlighting}[]
\KeywordTok{varImpPlot}\NormalTok{(randomf, }\DataTypeTok{main=}\StringTok{"Variables Random Forest"}\NormalTok{)}
\end{Highlighting}
\end{Shaded}

\begin{center}\includegraphics{PEC3_ML_files/figure-latex/unnamed-chunk-66-1} \end{center}

\hypertarget{predicciuxf3n-y-evaluaciuxf3n-del-algoritmo-4}{%
\subsubsection{Predicción y Evaluación del
algoritmo}\label{predicciuxf3n-y-evaluaciuxf3n-del-algoritmo-4}}

Creamos para ello la matriz de los resultados obtenidos en la predicción
del diagnóstico:

\begin{Shaded}
\begin{Highlighting}[]
\KeywordTok{library}\NormalTok{(caret)}
\NormalTok{predict.rf<-}\StringTok{ }\KeywordTok{predict}\NormalTok{(randomf, datos.test)}
\NormalTok{rf.matrix <-}\StringTok{ }\KeywordTok{confusionMatrix}\NormalTok{(datos.test}\OperatorTok{$}\NormalTok{diagnosis, predict.rf, }\DataTypeTok{positive =} \StringTok{"Maligno"}\NormalTok{)}
\NormalTok{rf.matrix}
\end{Highlighting}
\end{Shaded}

\begin{verbatim}
Confusion Matrix and Statistics

          Reference
Prediction Benigno Maligno
   Benigno     112       3
   Maligno       8      67
                                          
               Accuracy : 0.9421          
                 95% CI : (0.8988, 0.9707)
    No Information Rate : 0.6316          
    P-Value [Acc > NIR] : <2e-16          
                                          
                  Kappa : 0.8774          
                                          
 Mcnemar's Test P-Value : 0.2278          
                                          
            Sensitivity : 0.9571          
            Specificity : 0.9333          
         Pos Pred Value : 0.8933          
         Neg Pred Value : 0.9739          
             Prevalence : 0.3684          
         Detection Rate : 0.3526          
   Detection Prevalence : 0.3947          
      Balanced Accuracy : 0.9452          
                                          
       'Positive' Class : Maligno         
                                          
\end{verbatim}

Como vemos la precisión (Accuracy) es de 0.942 y Kappa es de 0.877.

\hypertarget{posible-mejora-aumento-del-nuxfamero-de-uxe1rboles}{%
\paragraph{Posible mejora: aumento del número de
árboles}\label{posible-mejora-aumento-del-nuxfamero-de-uxe1rboles}}

Veamos si incrementando el número de árboles el modelo mejora.
Entrenamos de nuevo el modelo:

\begin{Shaded}
\begin{Highlighting}[]
\KeywordTok{set.seed}\NormalTok{(params}\OperatorTok{$}\NormalTok{seed.clsfier)}
\NormalTok{rf}\FloatTok{.1000}\NormalTok{ <-}\StringTok{ }\KeywordTok{randomForest}\NormalTok{(diagnosis }\OperatorTok{~}\StringTok{ }\NormalTok{., }\DataTypeTok{data  =}\NormalTok{ datos.train, }\DataTypeTok{ntree=}\DecValTok{1000}\NormalTok{)}
\NormalTok{rf}\FloatTok{.1000}
\end{Highlighting}
\end{Shaded}

\begin{verbatim}

Call:
 randomForest(formula = diagnosis ~ ., data = datos.train, ntree = 1000) 
               Type of random forest: classification
                     Number of trees: 1000
No. of variables tried at each split: 5

        OOB estimate of  error rate: 2.9%
Confusion matrix:
        Benigno Maligno class.error
Benigno     238       4  0.01652893
Maligno       7     130  0.05109489
\end{verbatim}

\begin{Shaded}
\begin{Highlighting}[]
\KeywordTok{plot}\NormalTok{(rf}\FloatTok{.1000}\NormalTok{, }\DataTypeTok{main=}\StringTok{"modelo Random Forest 1000 árboles"}\NormalTok{)}
\end{Highlighting}
\end{Shaded}

\begin{center}\includegraphics{PEC3_ML_files/figure-latex/unnamed-chunk-68-1} \end{center}

\begin{Shaded}
\begin{Highlighting}[]
\NormalTok{predict.rf<-}\StringTok{ }\KeywordTok{predict}\NormalTok{(rf}\FloatTok{.1000}\NormalTok{, datos.test)}
\NormalTok{rf.matrix1000 <-}\StringTok{ }\KeywordTok{confusionMatrix}\NormalTok{(datos.test}\OperatorTok{$}\NormalTok{diagnosis, predict.rf,}\DataTypeTok{positive=}\StringTok{"Maligno"}\NormalTok{)}
\NormalTok{rf.matrix1000}
\end{Highlighting}
\end{Shaded}

\begin{verbatim}
Confusion Matrix and Statistics

          Reference
Prediction Benigno Maligno
   Benigno     112       3
   Maligno       7      68
                                          
               Accuracy : 0.9474          
                 95% CI : (0.9053, 0.9745)
    No Information Rate : 0.6263          
    P-Value [Acc > NIR] : <2e-16          
                                          
                  Kappa : 0.8888          
                                          
 Mcnemar's Test P-Value : 0.3428          
                                          
            Sensitivity : 0.9577          
            Specificity : 0.9412          
         Pos Pred Value : 0.9067          
         Neg Pred Value : 0.9739          
             Prevalence : 0.3737          
         Detection Rate : 0.3579          
   Detection Prevalence : 0.3947          
      Balanced Accuracy : 0.9495          
                                          
       'Positive' Class : Maligno         
                                          
\end{verbatim}

El nuevo modelo obtiene una precisión de 0.947 y una sensitividad y
especificidad de 0.958 y 0.941 respectivamente. Vemos que el modelo
obtenido con tres nodo tiene una mayor precision.

\hypertarget{paquete-caret-2}{%
\subsubsection{\texorpdfstring{Paquete
\emph{caret}}{Paquete caret}}\label{paquete-caret-2}}

Se vuelve a analizar el mismo dataset pero ahora usando el modelo
\texttt{rf} del paquete \texttt{caret}. Y lo validadmos con 5-fold cross
validation. Como sabemos, el fichero de train contiene 379 observaciones
y el de test 190.

\begin{Shaded}
\begin{Highlighting}[]
\CommentTok{## método : repeatedcv K-fold cross validation}
\CommentTok{## number : K folds}
\CommentTok{## repeats : número de repeticiones}
\KeywordTok{set.seed}\NormalTok{(params}\OperatorTok{$}\NormalTok{semilla)}
\NormalTok{ctrl2 <-}\StringTok{ }\KeywordTok{trainControl}\NormalTok{( }\DataTypeTok{method=}\StringTok{"repeatedcv"}\NormalTok{,}
                      \DataTypeTok{number=}\DecValTok{5}\NormalTok{,}
                      \DataTypeTok{summaryFunction =}\NormalTok{ defaultSummary,}
                      \DataTypeTok{verboseIter =}\OtherTok{FALSE}\NormalTok{,}
                      \DataTypeTok{repeats=}\DecValTok{3}\NormalTok{)}

\CommentTok{## Tunegrid para Random Forest}
\CommentTok{# mtry define cuántas variables se seleccionan al azar en cada split. Por }
\CommentTok{#  defecto sqrt(n.variables)}
\NormalTok{grid_rf <-}\StringTok{ }\KeywordTok{expand.grid}\NormalTok{(}\DataTypeTok{mtry =} \KeywordTok{c}\NormalTok{(}\DecValTok{2}\NormalTok{,}\DecValTok{4}\NormalTok{,}\DecValTok{8}\NormalTok{,}\DecValTok{16}\NormalTok{))}

\CommentTok{## trace <- FALSE para suprimir las iteraciones de entrenamiento}
\NormalTok{modelo.rf.caret     <-}\StringTok{ }\KeywordTok{train}\NormalTok{ (diagnosis }\OperatorTok{~}\StringTok{ }\NormalTok{.,}
                  \DataTypeTok{data =}\NormalTok{ datos.train,}
                  \DataTypeTok{method =}\StringTok{"rf"}\NormalTok{,}
                  \DataTypeTok{trControl=}\NormalTok{ctrl2,}
                  \DataTypeTok{tuneGrid =}\NormalTok{ grid_rf,}
                  \DataTypeTok{metric=}\StringTok{"Accuracy"}\NormalTok{,}
                  \DataTypeTok{prePoc =} \KeywordTok{c}\NormalTok{(}\StringTok{"center"}\NormalTok{, }\StringTok{"scale"}\NormalTok{),}
                  \DataTypeTok{verbose =}\OtherTok{FALSE}\NormalTok{,}
                  \DataTypeTok{trace =} \OtherTok{FALSE}\NormalTok{)}
\end{Highlighting}
\end{Shaded}

para la evaliuación del rendimiento creamos la matriz de resultados:

\begin{Shaded}
\begin{Highlighting}[]
\CommentTok{# Predicción del diagnóstico}
\NormalTok{predict.rf2 <-}\StringTok{ }\KeywordTok{predict}\NormalTok{(modelo.rf.caret, }\DataTypeTok{newdata =}\NormalTok{ datos.test)}
\CommentTok{# Matriz de Confusión}
\NormalTok{(cfrf <-}\StringTok{ }\KeywordTok{confusionMatrix}\NormalTok{(}\DataTypeTok{data=}\NormalTok{predict.rf2, datos.test}\OperatorTok{$}\NormalTok{diagnosis, }\DataTypeTok{positive=}\StringTok{"Maligno"}\NormalTok{))}
\end{Highlighting}
\end{Shaded}

\begin{verbatim}
Confusion Matrix and Statistics

          Reference
Prediction Benigno Maligno
   Benigno     111       7
   Maligno       4      68
                                          
               Accuracy : 0.9421          
                 95% CI : (0.8988, 0.9707)
    No Information Rate : 0.6053          
    P-Value [Acc > NIR] : <2e-16          
                                          
                  Kappa : 0.878           
                                          
 Mcnemar's Test P-Value : 0.5465          
                                          
            Sensitivity : 0.9067          
            Specificity : 0.9652          
         Pos Pred Value : 0.9444          
         Neg Pred Value : 0.9407          
             Prevalence : 0.3947          
         Detection Rate : 0.3579          
   Detection Prevalence : 0.3789          
      Balanced Accuracy : 0.9359          
                                          
       'Positive' Class : Maligno         
                                          
\end{verbatim}

La precisión es similar al modelo anterior sin el paquete
\texttt{caret}.

Accuracy es de 0.942\\
Kappa es de 0.878

\hypertarget{resumen-de-resultados}{%
\section{Resumen de resultados}\label{resumen-de-resultados}}

En la siguiente tabla se muestra un resumen con los principales
parámetros obtenidos de la función \texttt{confusionMatrix} del paquete
\texttt{caret} en cada uno de los algoritmos:

Ordenamos los datos según los valores de precisión y especificidad.

\begin{Shaded}
\begin{Highlighting}[]
\CommentTok{# Ordenados por precisión y sensibilidad}
\NormalTok{dplyr}\OperatorTok{::}\KeywordTok{arrange}\NormalTok{(resultado.final, Accuracy, Sensitivity)}
\end{Highlighting}
\end{Shaded}

\begin{verbatim}
                 modelo Accuracy Kappa Sensitivity Specificity
1      ANN caret 5-fold    0.921 0.835        0.91        0.93
2           ANN 3 nodos    0.926 0.847        0.93       0.922
3                    27    0.932 0.853        0.84       0.991
4        Arbol decision    0.932 0.856        0.89       0.957
5     Arbol decision 10    0.932 0.856        0.91       0.948
6             ANN caret    0.932 0.857        0.92       0.939
7          knn z transf    0.937 0.864        0.84           1
8            SVM Lineal    0.937 0.866        0.88       0.974
9             SVM Caret    0.937 0.866        0.88       0.974
10      SVM Caret 5fold    0.937 0.866        0.88       0.974
11                    1    0.937 0.867       0.907       0.957
12           ANN 1 nodo    0.937 0.869        0.95        0.93
13                   17    0.942 0.876       0.867       0.991
14                   23    0.942 0.876       0.867       0.991
15                  knn    0.942 0.876        0.87       0.991
16  Random forest caret    0.942 0.878        0.91       0.965
17        Random forest    0.942 0.877        0.96       0.933
18                    5    0.947 0.889       0.907       0.974
19   Random forest 1000    0.947 0.889        0.96       0.941
20 Arbol decision caret    0.953   0.9        0.91       0.983
21                   11    0.958  0.91       0.893           1
22            SVM Gauss    0.958 0.911        0.93       0.974
23      SVM Caret boots    0.958 0.911        0.93       0.974
24       Naive-Bayes L1    0.963 0.922        0.93       0.983
25       Naive-Bayes L0    0.968 0.934        0.95       0.983
\end{verbatim}

\begin{Shaded}
\begin{Highlighting}[]
\NormalTok{better.results <-}\StringTok{ }\NormalTok{dplyr}\OperatorTok{::}\KeywordTok{filter}\NormalTok{(resultado.final, Accuracy }\OperatorTok{>}\StringTok{ }\FloatTok{0.946}\NormalTok{)}
\NormalTok{results <-}\StringTok{ }\NormalTok{dplyr}\OperatorTok{::}\KeywordTok{arrange}\NormalTok{(better.results, Accuracy, Sensitivity)}
\NormalTok{results}
\end{Highlighting}
\end{Shaded}

\begin{verbatim}
                modelo Accuracy Kappa Sensitivity Specificity
1                    5    0.947 0.889       0.907       0.974
2   Random forest 1000    0.947 0.889        0.96       0.941
3 Arbol decision caret    0.953   0.9        0.91       0.983
4                   11    0.958  0.91       0.893           1
5            SVM Gauss    0.958 0.911        0.93       0.974
6      SVM Caret boots    0.958 0.911        0.93       0.974
7       Naive-Bayes L1    0.963 0.922        0.93       0.983
8       Naive-Bayes L0    0.968 0.934        0.95       0.983
\end{verbatim}

En la Tabla anterior se pueden observar que los clasificadores
Naive-Bayes fueron los más acertados a la hora de clasificar las
muestras de los pacientes, obteniendo un 96\% de precisión. Por el
contrario, se puede observar que los clasificadores ANN con el paquete
\texttt{caret} y Árboles de decisión clasifican las muestras con una
precisón del 89 \% y 93\% en promedio respectivamente.

\hypertarget{discusiuxf3n-y-conslusiones}{%
\section{Discusión y conslusiones}\label{discusiuxf3n-y-conslusiones}}

El mejor modelo a aplicar dependerá de la importancia que le demos tanto
a la precisión como a la sensibilidad y especificidad en la detección
del cáncer de mama según nuestros resultados.

Como podemos observar en el siguiente gráfico de barras todos los
modelos generados tienen una accuracy por encima del 85\% siendo el
modelo generado por el algoritmo Naive-Bayes con Laplace 0 el que mayor
potencia de clasificación tiene con una accuracy del 96.8\%.

\begin{Shaded}
\begin{Highlighting}[]
\KeywordTok{ggplot}\NormalTok{(}\DataTypeTok{data=}\NormalTok{resultado.final, }\KeywordTok{aes}\NormalTok{(}\DataTypeTok{x=}\NormalTok{modelo, }\DataTypeTok{y=}\NormalTok{Accuracy, }\DataTypeTok{fill=}\NormalTok{modelo)) }\OperatorTok{+}\StringTok{ }
\StringTok{    }\KeywordTok{geom_bar}\NormalTok{(}\DataTypeTok{stat=}\StringTok{"identity"}\NormalTok{, }\DataTypeTok{position=}\StringTok{"dodge"}\NormalTok{) }\OperatorTok{+}\StringTok{  }\KeywordTok{theme}\NormalTok{(}\DataTypeTok{text =} \KeywordTok{element_text}\NormalTok{(}\DataTypeTok{size=}\DecValTok{8}\NormalTok{),}
                                                         \DataTypeTok{legend.position=}\StringTok{'bottom'}\NormalTok{) }
\end{Highlighting}
\end{Shaded}

\begin{center}\includegraphics{PEC3_ML_files/figure-latex/unnamed-chunk-94-1} \end{center}

La sensibilidad caracteriza la capacidad de la prueba para detectar la
enfermedad en sujetos enfermos, lo que tras la precisión quizá debería
de ser el factor a tener en cuenta y ante igualdad, un mejor valor de
especificidad. El modelo con mejor precisión es \emph{Naive-Bayes L0}
sin embargo hay otros modelos con mejor sensibilidad y valores parecidos
de precisión. \emph{Random forest} tiene una elevada precisión y
sensibilidad pero un valor muy bajo de \texttt{kappa}. \emph{Arbol
decision caret} tiene elevada Sensibilidad y pero menor percisión y
specificidad.

Para este conjunto de datos, los modelos que han tenido menos potencia
de clasificación han sido los generados por el algoritmo de Artificial
Neural Network (ANN) del paquete \texttt{caret}, por debajo del 90\% de
accuracy.

El algoritmo de árboles de decisión y Random Forest han tenido valores
intermedios de Accuracy, sin embargo con valores elevados de
Sensibility.

He encontrado diferencias significativas a la hora de generar el modelo
con paquetes específicos de cada algoritmo o utilizando el paquete
\texttt{caret}, especialmente con el algoritmo SVM que mejora
significamente si empleamos algún método como 5-fold validation o
boostrapping. En cambio con el algoritmo ANN los resultados con
\texttt{caret} muestran menos potencia.

Como conslusión, según estos resultados y los objetivos que buscamos con
nuestros análisis, el modelo \emph{Naive-Bayes L0} sería el más
apropiado, con un porcentaje cercano al 95\% tanto en precisión como en
sensibilidad y con valores de kappa y especificidad aceptables, lo que
podemos apreciar observando el siguiente gráfico donde podemos
compararlo con los otros modelos que han mostrado mejores resultados:

\begin{Shaded}
\begin{Highlighting}[]
\KeywordTok{ggplot}\NormalTok{(}\DataTypeTok{data=}\NormalTok{results, }\KeywordTok{aes}\NormalTok{(}\DataTypeTok{x=}\NormalTok{modelo, }\DataTypeTok{y=}\NormalTok{Accuracy, }\DataTypeTok{fill=}\NormalTok{Sensitivity)) }\OperatorTok{+}\StringTok{ }
\StringTok{    }\KeywordTok{geom_bar}\NormalTok{(}\DataTypeTok{stat=}\StringTok{"identity"}\NormalTok{, }\DataTypeTok{position=}\StringTok{"dodge"}\NormalTok{) }\OperatorTok{+}\StringTok{  }\KeywordTok{theme}\NormalTok{(}\DataTypeTok{text =} \KeywordTok{element_text}\NormalTok{(}\DataTypeTok{size=}\DecValTok{8}\NormalTok{),}
                                                         \DataTypeTok{legend.position=}\StringTok{'bottom'}\NormalTok{) }
\end{Highlighting}
\end{Shaded}

\begin{center}\includegraphics{PEC3_ML_files/figure-latex/unnamed-chunk-95-1} \end{center}

\hypertarget{bibliografuxeda}{%
\section*{Bibliografía}\label{bibliografuxeda}}
\addcontentsline{toc}{section}{Bibliografía}

\hypertarget{refs}{}
\leavevmode\hypertarget{ref-core2013r}{}%
Core Team, RCTR, and others. 2013. ``R: A Language and Environment for
Statistical Computing.'' \emph{R Foundation for Statistical Computing,
Vienna}.

\leavevmode\hypertarget{ref-github2016github}{}%
Github, Inc. 2016. ``GitHub.''

\leavevmode\hypertarget{ref-kuhn2017caret}{}%
Kuhn, Max. 2017. ``A Short Introduction to the Caret Package.'' \emph{R
Foundation for Statistical Computing, Vienna, Austria.}
\url{https://cran.r-project.org/web/packages/caret/vignettes/caret.pdf}.

\leavevmode\hypertarget{ref-lantz2015machine}{}%
Lantz, Brett. 2015. \emph{Machine Learning with R}. Packt Publishing
Ltd. \url{http://www.packtpub.com/books/content/machine-learning-r}.

\end{document}
